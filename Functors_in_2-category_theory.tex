\section{Functors in 2-category theory}

The reference for this section is \cite[2.2.4-2.2.8]{kerodon}.

\begin{defi}
\label{lax_functor_defi}
	Let $ \mathcal{ C } , \mathcal{ D } $ be 2-categories.
	A \textbf{lax 2-functor} $ F \colon \mathcal{ C } \to \mathcal{ D } $ consists of the following data:
	\begin{itemize}
		\item 
		A map $ F \colon \Ob ( \mathcal{ C } ) \to \Ob ( \mathcal{ D } ) $ where $ X \mapsto F ( X ) $,
		
		\item 
		for all $  X , Y \in \Ob ( \mathcal{ C } ) $ a functor $ F = F_{ X , Y } \colon \underline{ \Hom }_{ \mathcal{ C } } ( X , Y ) \to \underline{ \Hom }_{ \mathcal{ D } } ( F X , F Y ) $,
		
		\item 
		for all $ X , Y , Z \in \Ob( \mathcal{ C } ) $ a morphism $ \epsilon_X \colon \id_{ F X } \Rightarrow F ( \id_X ) $ in $ \underline{ \Hom }_{ \mathcal{ D } } ( F X , F X ) $ called \textbf{unit constraint},
		
		\item 
		for all $ X , Y , Z \in \Ob ( \mathcal{ C } ) $ a commutative diagram
		\[
		\begin{tikzcd}
			\underline{ \Hom }_{ \mathcal{ C } } ( Y , Z ) \times 
			\underline{ \Hom }_{ \mathcal{ C } } ( X , Y )
			\ar[r, " - \circ - " ]
			\ar[r, shift right=2em, Rightarrow, " \mu "]
			\ar[d, " F_{ Y , Z } \times F_{ X , Y } "]
			&
			\Hom_{ \mathcal{ C } } ( X , Z )
			\ar[d, " F_{ X , Z }"]
			\\
			\underline{ \Hom }_{\mathcal{ D } }( F Y , F Z ) \times 
			\Hom_{ \mathcal{ D } } ( F X , F Y )
			\ar[r, " - \circ - "]
			&
			\underline{ \Hom }_{ \mathcal{ D } } ( F X , F Z )
		\end{tikzcd}
		\] 
		where $ \mu_{ g , f } \colon F ( g ) \circ F ( f ) \Rightarrow F ( g \circ f ) $ is called the \textbf{composition constraint}.
		The above data is required to satisfy the following:
		\begin{enumerate}[label=(\alph*)]
			\item 
				For all $ f \colon X \to Y $ 1-morphisms in $ \mathcal{ C } $ the following diagrams commute
				\[
				\begin{tikzcd}
					F ( \id_Y ) \circ F ( f ) 
					\ar[r, Rightarrow, "\mu"]
					&
					F ( \id_Y \circ f ) 
					\ar[d, Rightarrow, " F ( \lambda_f )"]
					\\
					\id_{FY} \circ Ff
					\ar[u, Rightarrow, "\epsilon_Y * \id_{ F ( f ) }" ]
					\ar[r , Rightarrow , "\lambda_{ F ( f ) }"]
					&
					F ( f ) 
				\end{tikzcd}	
				\quad
				\begin{tikzcd}
					F ( f ) \circ F ( \id_X ) 
					\ar[r, Rightarrow, "\mu"]
					&
					F ( f \circ \id_X ) 
					\ar[d, Rightarrow, " F ( \rho _f )"]
					\\
					\id_{ F Y } \circ Ff
					\ar[u, Rightarrow, " \id_{ F ( f ) } * \epsilon_X " ]
					\ar[r , Rightarrow , "\rho_{ F ( f ) }"]
					&
					F ( f ) 
				\end{tikzcd}	
				\]
				in $\underline{\Hom} ( FX , FY ) $,
				
				\item 
				and for all $ W , X , Y , Z \in \Ob ( \mathcal{ C } ) $ and for all $ W \xrightarrow{ f } X \xrightarrow{ g } Y \xrightarrow{ h } Z $ 1-morphisms in $ \mathcal{ C } $ the following diagram commutes: 
				\[
				\begin{tikzcd}
					F ( h ) \circ ( F ( g ) \circ F ( f ) )
					\ar[r, Rightarrow, "\alpha^{ \mathcal{ D } }"]
					\ar[d, Rightarrow, " \id_{Fh} * \mu "]
					&
					( F ( h ) \circ F ( g ) ) \circ F ( f )
					\ar[d, Rightarrow, " \mu * \id_{ F ( f ) } "]
					\\
					F ( h ) \circ F ( g \circ f )
					\ar[d , Rightarrow, "\mu "]
					&
					F ( h \circ g ) \circ F ( f )
					\ar[d, Rightarrow, "\mu"]
					\\
					F ( h \circ ( g \circ f ) )
					\ar[r, Rightarrow, " F ( \alpha^{ \mathcal{ C } } ) "]
					&
					F ( ( h \circ g ) \circ f )
				\end{tikzcd}
				\]
				in $ \underline{ \Hom }_{ \mathcal{ D } } ( F W , F Z )$.
 		\end{enumerate}
	\end{itemize}
\end{defi}

\begin{defi}
\label{twofunctor_defi}
	A \textbf{2-functor} $ F \colon \mathcal{ C } \to \mathcal{ D } $ is a lax 2-functor such that for all $ X \in \Ob ( \mathcal{ C } ) $ 
	the morphism $ \epsilon_X \colon \id_{ F X } \xRightarrow{\sim} F ( \id_X )$ is invertible and such that $ \forall X , Y \in \Ob (\mathcal{ C } ), \forall X \xrightarrow{ f } Y \xrightarrow{ g } Z $ the morphism $ \mu_{ g , f } \colon F ( g ) \circ F ( f )
	\xRightarrow{ \sim } F ( g \circ f ) $ is invertible.
\end{defi}

\begin{defi}
\label{strict_twofunctor_defi}
	A \textbf{strict 2-functor} is a 2-functor, such that for all $ X \in \Ob ( \mathcal { C } )$ the following hold $ \epsilon_X = \id \colon \id_{ FX } \Rightarrow F ( \id_X ), \forall X , Y , Z \in \Ob ( \mathcal{ C } )$ and for all composable morphisms $ X \xrightarrow{ f } Y \xrightarrow{ g } Z $ and $ \mu_{ g , f } = \id = F ( g ) \circ F ( f ) \Rightarrow F ( g \circ f ) $.
\end{defi}

\begin{exmp}
	Lax monoidal functors $ \mathcal{ M } \to \mathcal{ N } $ for $ \mathcal{ M } $ and $ \mathcal{ N } $ monoidal categories correspond to lax 2-functors $ B \mathcal{ M } \to B \mathcal{ N } $.
\end{exmp}

\begin{exmp}
	Let $ S $ be a set and $ \mathcal{ E }_S $ be a category with $ \Ob ( \mathcal{ E }_S ) = S $ and for all $ x , y \in S, \underline{ \Hom }_{ \mathcal{ E }_S } ( x , y ) \coloneqq \{ *
	\} $.
	\begin{itemize}
		\item 	
		Fix $ \mathcal{ M } $ a monoidal category and let $ B \mathcal{ M } $ be its delooping and $ \underline{ \mathcal{ C } } \colon \mathcal{ E }_S \to B \mathcal{ M }$ a lax monoidal functor.
		
		\item 
		Fix a map $ \underline{ \mathcal{ C } } \colon \Ob ( \mathcal{ E }_S )= S \to \Ob ( B \mathcal{ M } ) = \{ * \}$.
		
		\item 
		For all $ X , Y \in \Ob ( \mathcal{ E }_S ) = S $ a functor: 
		\begin{align*}
				\underline{ \Hom }_{ \mathcal{ E }_S } ( X , Y )
				&\to
				\mathcal{ M } = \underline{ \Hom }_{ B \mathcal{ M } } ( * , * )
				\\
				*
				&\mapsto
				\underline{ \mathcal{ C } } ( X  , Y )
		\end{align*} 
		
		\item 
		For all $ X  \in S $, $ \epsilon_X \colon \id_{ \underline{ \mathcal{ C } } ( X ) } = \id_*$, such that 
		\[
		\begin{tikzcd}
			\mathcal{ M } = \underline{ \Hom }_{ B \mathcal{ M } } ( * , * )
			\ar[loop below, " \mathds{ 1 }_{ \mathcal{ M } } = \id_* "]
			&
			\mathds{ 1 }_{ \mathcal{ M } } \otimes M 
			\ar[r, "\sim", "\lambda_M"']
			&
			M
		\end{tikzcd}
		\]
		This is called the identity constraint.
		
		\item 
		For all $ X , Y , Z \in \Ob ( \mathcal{ E }_S = S ) $
		\[
		\begin{tikzcd}
			\{ * \} \times \{ * \} = \underline{\Hom}_{\mathcal{ E }_S } ( Y , Z ) \times \underline{ \Hom }_{ \mathcal{ E }_S } ( X , Y )
			\ar[r, "\sim"', "\circ"]
			\ar[r,shift right =2em, Rightarrow, "\mu"]
			\ar[d, "\underline{ \mathcal{ C } } \times \underline{ \mathcal{ C } }"]
			&
			\underline{\Hom}_{\mathcal{E}_S}( X , Z ) = \{ * \}
			\ar[d, "\underline{\mathcal{C}}"]
			\\
			\underline{\Hom}_{B\mathcal{ M } } ( * , * ) \times
			\underline{\Hom}_{B \mathcal{ M } } ( * , * )
			\ar[r, "\otimes"]
			&
			\Hom_{B\mathcal{ M } } ( * , * )
		\end{tikzcd}
		\]
		where $ \mu \colon \underline{\mathcal{ C }}( Y , Z ) \otimes \underline{ \mathcal{ C } } ( X , Y ) \to \underline{ \mathcal{ C } ( X , Z ) }$.
		The above data should satisfy the following:
		$ \forall X , Y \in \Ob ( \mathcal{ E }_S ) = S $.
		\[
		\begin{tikzcd}
			\underline{ \mathcal{ C } } ( X , Y ) \otimes \underline{ \mathcal{ C } } ( X , Y ) 
			\ar[r, "\mu"]
			&
			\underline{ \mathcal{ C } } ( X , Y )
			\ar[d, " \id "]
			\\
			\mathds{ 1 }_{ \mathcal{ M } } \otimes \underline{ \mathcal{ C } } ( X , Y )
			\ar[r, " \lambda_{ \underline{\mathcal{ C }} ( X , Y )} "]
			\ar[u, " \mathcal{ E }_Y \otimes \id"]
			&
			\underline{ \mathcal{C} }( X , Y )
		\end{tikzcd}
		\]
		
		\item 
		Let $ \mathcal{ M } = \Set $
		\[
		\begin{tikzcd}
			( \id_X , f )
			\ar[r, mapsto]
			&
			\id_X \circ f 
			\ar[d, equal]
			\\
			( x , f )
			\ar[u , mapsto]
			\ar[r, mapsto]
			&
			f \colon X \to Y
		\end{tikzcd}
		\]
		similarly for the right constraint.
		
		\item 
		For all $ W , X , Y , Z \in \Ob ( \mathcal{E }_S ) $ the following diagram commutes:
		\[
		\begin{tikzcd}
			\underline{\mathcal{ C } } ( Y , Z ) \otimes  ( \underline{\mathcal{ C } } ( X , Z ) \otimes \underline{\mathcal{ C } } ( W , X ) )
			\ar[d]
			\ar[r, " \alpha^M " ]
			&
			( \underline{\mathcal{ C } } ( Y , Z ) \otimes   \underline{\mathcal{ C } } ( X , Z ) ) \otimes \underline{\mathcal{ C } } ( W , X ) 
			\ar[d]
			\\
			\underline{ \mathcal{ C } } ( Y , Z ) \otimes 
			\underline{ \mathcal{ C } } ( W , Y )
			\ar[d, " \mu" ]
			&
			\underline{\mathcal{ C } } ( X , Z ) \otimes
			\underline{\mathcal{ C } } ( w , Z )
			\ar[d]
			\\
			\underline{ \mathcal{ C } } ( W , Z )
			\ar[r, " \underline{\mathcal{ C}} ( \alpha ) = \id "]
			&
			\underline{\mathcal{ C } } ( W , Z )
		\end{tikzcd}
		\] 
	\end{itemize}
\end{exmp}

\begin{rmk}
	Dg-categories are Lax 2-functors.
\end{rmk}	

\begin{Exercise}
	Let $ \mathcal{ C } $ be a 2-category and $ \mathcal{ D } $ a 1-category, then every lax 2-functor $ F \colon  \mathcal{ C } \to \mathcal{ D } $ is strict.
\end{Exercise}

\begin{defi}
	A $ F \colon \mathcal{ C } \to \mathcal{ D } $ lax functor between 2-categories is 
	\begin{itemize}
		\item 
		\textbf{unitary} if $  \forall X \in \mathcal{ C } , \epsilon_X $ is an isomorphism,
		
		\item 
		\textbf{strictly unitary} $ \forall X \in \mathcal{ C } , \epsilon_X = \id_X $,
		
		\item 
		composition of lax functors
		\[
			\mathcal{ C } \xrightarrow{ F } 
			\mathcal{ D } \xrightarrow{ G } 
			\mathcal{ E }
		\]  
		between 2-categories is the lax functor $ G \circ F $ defined as follows:
		
		\begin{itemize}
			\item 
			on objects $( G \circ F ) ( X ) = G ( F ( X ) )$,
			
			\item 
			on morphisms the composition is given by the dashed arrow,
			\[
			\begin{tikzcd}
				\underline{ \Hom }_{ \mathcal{ C } } ( X , Y )
				\ar[r, dashed, "GF_{ X , Y } "]
				\ar[d, " F "]
				&
				\underline{ \Hom }_{ \mathcal{ E } } ( ( G F ) X , ( G F ) Y )
				\ar[d, equal]
				\\
				\underline{ \Hom }_{ \mathcal{ D } } ( F X , F Y )
				\ar[r, "G_{ F X , F Y }"]
				&
				\underline{\Hom}_{\mathcal{ E } } ( G ( F X ), G ( F Y ) )   
			\end{tikzcd}
			\]
			
			\item 
			identity constraints are given as,
			\[
			\begin{tikzcd}
				&G ( \id_{ F X } )
				\ar[rd, Rightarrow, " G ( \epsilon_X^F) "]
				\\
				\id_{ G F X }
				\ar[ru, Rightarrow, " \epsilon_{ F X }^G"]
				\ar[rr, Rightarrow, equal, " \epsilon_X^{ G F }"']
				&&
				GF ( id_X )
			\end{tikzcd}
			\quad 
			\begin{tikzcd}		
				\id_{ F X }
				\ar[d, Rightarrow, "\epsilon_X^F"]
				\\
				F ( \id_X ) 
			\end{tikzcd}
			\]
			
			\item 
			$ \forall X \xrightarrow{ f } Y \xrightarrow{ g } Z$ 
			1-morphisms in $ \mathcal{ C }$ the bottom arrow in the following diagram gives the composition constraint:
			\[
			\begin{tikzcd}
				&
				G ( F ( g ) \circ F ( f ) )
				\ar[rd, Rightarrow, " G ( \mu_{ g , f }^F) "]
				\\
				GF ( g ) \circ GF ( f ) 
				\ar[ru, Rightarrow, " \mu_{ F g , F f }^G"]
				\ar[rr, Rightarrow, equal, " \mu_{ g , f }^{ G F }"']
				&&
				GF ( g \circ f )
			\end{tikzcd}
			\]
		\end{itemize}
	\end{itemize}
\end{defi}

