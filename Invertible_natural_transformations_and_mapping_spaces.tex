\section{Invertible natural transformations and mapping spaces}

\begin{defi}
	Let $ \mathcal{ C } $ be an $ \infty $-category, $ X , Y \in \mathcal{ C }_0 $ objects of $ \mathcal{ C } $.
	The simplicial set $ \mathcal{ C } ( X , Y ) $ is defined via the pullback
	\[
	\begin{tikzcd}
		\mathcal{ C } ( X , Y )
		\ar[r]
		\ar[d]
		&
		\twoHom_{ \Set_\Delta } ( \Delta^1 , \mathcal{ C } )
		\ar[d, shift left, "i"]
		\ar[d, shift right = 3em, "{ ( s , t ) }"'] 
		\\
		\Delta^0
		\ar[r, " { ( x , y ) } "]
		&
		\mathcal{ C } \times \mathcal{ C }
		\cong
		\twoHom ( \partial \Delta^1 , \mathcal{ C } ) 
	\end{tikzcd}
	\]
\end{defi}

Why is $ \mathcal{ c } ( X , Y ) $ a Kan complex ($\infty$-groupoid)?
To answer the question we take a detour and study invertible natural transformations.

\begin{Motivation}
	Let $ \mathcal{ C } , \mathcal{ D } $ be 1-categories and $ F , G \colon \mathcal{ C } \to \mathcal{ D } $ functors together with a natural transformation $ \eta \colon F \to G $, this yields the following commutative diagram:
	\[
	\begin{tikzcd}
		\{ 1 \} \times \mathcal{ C }
		\ar[d]
		\ar[rd, bend left , " G" ]
		\\
		\left[ 1 \right] \times \mathcal{ C } \ar[ r , " \overline{\eta} "]
		&
		\mathcal{ D } 
		\\
		\{ 0 \} \times \mathcal{ C } 
		\ar[ru, bend right, " F "']
		\ar[u]
	\end{tikzcd}
	\]
\end{Motivation}