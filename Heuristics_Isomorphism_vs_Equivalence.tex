\section{Heuristics: Isomorphism vs. Equivalence}

Makkai's Principle of Isomorphism (1998) says:

"All grammatically correct properties about objects in a fixed category are to be invariant under isomorphism."

Fix a category $ \mathcal{ C } $ and a small category $ A $ and take the functor category $ \Fun ( A , \mathcal{ C } )$, that is $ A $-shaped diagrams in $ \mathcal{ C } $.
For $ X \in \mathcal{ C } $ define:
\[
    \lim_{a \in A} \Hom_{\mathcal{ C }} ( X , D(a)) \coloneqq \{ ( p_a \colon X \to  D ( a ) )_{ a \in A } \mid 
    \begin{tikzcd}
    X
    \ar[d, "f_a"']
    \ar[rd, "f_b"]
    \\
    D ( a ) 
    \ar[r]
    &
    D ( b ) 
    \end{tikzcd}
    \forall f \colon a \to b \text{ in } A
    \}
\]

For $ \phi \colon Y \to X $ define the function 
\begin{align*}
    \phi^* \colon \lim_{a \in A} \Hom_{ \mathcal{ C }} ( X , D ( a ) ) 
    &\to
    \lim_{ a \in A } \Hom_{ \mathcal{ C }} ( Y , D ( a ) ) 
    \\
    p = ( p_a \colon X \to D ( a ) )_{ a \in A }
    &\mapsto
    \phi^* ( p ) = ( p_a \circ \phi \colon Y \to D ( a ) )_{ a \in A }
\end{align*}

Thus $ \lim_{a \in A } \Hom_{ \mathcal{ C } } ( - , D ( a ) ) \colon \mathcal{ C }^{\op} \to \Set $ is a presheaf of sets on $ \mathcal{ C } $.

\begin{defi}
    A limit of a diagram $ D \colon A \to \mathcal{ C } $ is a cone $ p \in \lim_{a \in A } \Hom_{ \mathcal{ C } } ( X, D ( a ) ) $ that is universal in the sense that $ \forall Y \in \mathcal{ C }, \forall q \in \lim_{a \in A} \Hom_{\mathcal{ C }} ( Y , D ( a ) ) \exists! \varphi\colon Y \to X $ such that $ \varphi^* ( p ) = q $.
    We write $ \lim_{ a \in A } D ( a ) $ for any limit of $ D $ (which may or may not exist).
\end{defi}

\begin{rmd}[Yoneda Lemma]
    Let $ X \in \mathcal{ C } $ and let 
    \begin{align*}
        \nu \colon \Nat ( \Hom_{\mathcal{ C } } ( - , X ) , \lim_{ a \in A } \Hom_{\mathcal{ C } } ( - , D ( a ) ) ) 
        &\isomorphism
        \lim_{ a \in A } \Hom_{ \mathcal{ C } }( X , D ( a ) ) 
        \\
        \eta = ( \eta_Y \colon \Hom_{ \mathcal{ C } } (  Y , X ) \to \lim_{ a \in \mathcal{ C } } ( Y , D ( a ) )_{ Y \in \mathcal{ C } } 
        &\mapsto 
        \eta_X ( \id_X )
        \\
        ( \eta^p_Y ( p ) \coloneqq \varphi^* ( p ) )_{ Y \in \mathcal{ C }}
        &\mapsfrom
        p
    \end{align*}
\end{rmd}

If $ p \in \lim_{a \in A} \Hom_{ \mathcal{ C } } ( X , D ( a ) ) $ is a limit of $ D \colon A \to \mathcal{ C } $ then 

$ \eta^p \colon \Hom_{ \mathcal{ C } } ( - ,  X ) \isomorphism \Hom_{ \mathcal{ C } } ( - , D ( a ) ) $ is a natural isomorphism. 
We furthermore obtain, that for an isomorphism $\psi \colon Y \to X $ in $\mathcal{ C } $ we have 
\[
\begin{tikzcd}
    & 
    \Hom_{\mathcal{ C } } ( - , X ) 
    \ar[rd, " \eta^p ", "\sim"']
    \\
    \Hom_{ \mathcal{ C } } ( - , Y ) 
    \ar[ru , " \psi^*" , " \sim"' ]
    \ar[rr, " \sim " ]
    &&
    \lim_{ a \in A } \Hom_{ \mathcal{ C } } ( - , D ( a ) ) 
\end{tikzcd}
\]
\begin{exmp}
    Let the following be a diagram in $ \mathcal{ C } $ 
    \[
    \begin{tikzcd}
        &
        Y
        \ar[d , "g" ]
        \\
        X
        \ar[r, "f" ]
        &
        Z
    \end{tikzcd}
    \]
    then the limit of the diagram (if it exists ) is called a pullback.
    \[
    \begin{tikzcd}
        W
        \ar[rrd, bend left, "\forall f'' " ]
        \ar[rd, " \exists! p "]
        \ar[rdd, bend right, " \forall g'' "' ]
        \\
        &
        X \times_Z Y 
        \ar[ r , " f'" ]
        \ar[d, " g' "]
        &
        Y
        \ar[d, " g "]
        \\
        &
        X
        \ar[r, "f"]
        &
        Z
    \end{tikzcd}
    \]
    For example if $ \mathcal{ C } = \Set$ then $ X  \times_Z Y = \{ ( X , Y ) \in X \times Y \mid f ( x ) = g ( y ) \}$.
\end{exmp}

The commutativity condition takes place in $ \Hom_{ \mathcal{ C } } ( X \times_Z Y , Z ) \ni g \circ f' = f \circ g' $

\underline{ Makkai's Pronciple of Equivalence }
All grammatically correct properties of objects in a fixed 2-category are to be invariant under equivalence.

\begin{rmk}
    We want to $ \Cat $ be the strict 2-category of (small) categories with functors as 1-morphisms and natural transformations as 2-morphisms.
    Now natural transformation allow for a notion of equivalence of morphisms, that is in a 1-category we only knew what it meant for two morphisms to be equal, but now we can talk about two functors being naturally isomorphic given us a notion of equivalence of 1-morphisms, via the 2-morphisms.
\end{rmk}

\begin{defi/prop}[Godement Product]
    Consider natural transformations 
    \[
    \begin{tikzcd}
        &
        {}
        \ar[dd, "\alpha"]
        &&
        {}
        &
        \\
        \mathcal{ C }
        \ar[rr, bend left , "F_1"]
        \ar[rr, bend left , "F_2"]
        &&
        \mathcal{ D }
        &&
        \mathcal{ E }
        \\
        &
        {}
        &&
        {}
        &
    \end{tikzcd}
    \]
    \todo{ how to typeset this }
    Their Godement product is the natural transformation.
    Let $ X \in \mathcal{ C } $, we obtain the following diagram
    \[
    \begin{tikzcd}
        F_1 ( X ) 
        \ar[d, " \alpha_x "]
        &
        G_1 ( F_1 ( X ) ) 
        \ar[d, " G_1 ( \alpha_X ) "']
        \ar[rd, " {( \beta * \alpha )_X} "]
        \ar[r , " \beta_{ F_1 ( X ) } "]
        & 
        G_2 ( F_1 ( X ) )
        \ar[d, " G_2 ( \alpha_X ) "]
        \\
        F_2 ( X ) 
        &
        G_1 ( F_2 ( X ) ) 
        \ar[ r, " \beta_{ F_2 ( X ) }"]
        &
        G_2 ( F_2 ( X ) ) 
    \end{tikzcd}    
    \]
    in $ \mathcal{ D } $.
\end{defi/prop}

\begin{proof}
    We show that $ \beta * \alpha \colon G_1 \circ F_1 \Rightarrow g_2 \circ F_2 $ is indeed a natural transformation. 
    For that we take the following diagram 
     \[
     \begin{tikzcd}
         X
         \ar[d, "f"]
         &
         G_1 ( F_1 ( X ) ) 
         \ar[r , " G_1 ( \alpha_X ) "]
         \ar[r, " G_1 ( \alpha_X ) "]
         \ar[d, " G_1 ( F_1 ( f ) ) "']
         &
         G_1 ( F_2 ( X ) ) 
         \ar[ r, " \beta_{ F_2 ( X ) } "]
         \ar[d, " G_1 ( F_2 ( f ) ) "]
         & 
         G_2 ( F_2 ( X ) )
         \ar[d, " G_2 ( F_2 ( f ) ) "]
         \\
         Y
         &
         G_1 ( F_1 ( Y ) )
         \ar[ r , " G_1 ( \alpha_1 ) "']
         &
         G_1 ( F_2 ( Y ) ) 
         \ar[ r, " {\beta_{ F_2 ( Y ) }}"']
         & 
         G_2 ( F_2 ( Y ) ) 
     \end{tikzcd}
     \]
\end{proof}

\begin{prop}
    Consider natural transformations 
    \todo{missing}
    Then $ ( \delta \beta ) * ( \gamma \alpha ) = ( \delta * \gamma ) \circ ( \beta * \alpha )$.
\end{prop}

\begin{proof}
    Let $ X \in \mathcal{ C } $ 
    \[
    \begin{tikzcd}
        G_1 ( F_1 ( X ) ) 
        \ar[r, " \beta_ {F_1 ( X ) }"]
        \ar[rd, " ( \beta * \alpha )_X "]
        \ar[d, " G_1 (\alpha_X ) "']
        \ar[dd, bend right=90, " G_1 ( \gamma_X \alpha_X ) "']
        &
        G_2 ( F_1 ( X ) ) 
        \ar[d, " G_2 ( \alpha_X ) "]
        \ar[r, " \delta_{ F_1 ( X ) }"]
        &
        G_3 ( F_1 ( X ) ) 
        \ar[dd, bend left =90, " G_3 ( \gamma_X \alpha_X ) "]
        \ar[ d , " G_3 ( \alpha_X ) "]
        \\
        G_1 ( F_2 ( X ) ) 
        \ar[d, " G_1 ( \gamma_X )"']
        \ar[r, " \beta_{ F_2 ( X ) }"]
        &
        G_2 ( F_2 ( X ) ) 
        \ar[r, " \delta_{ F_2 ( X ) }"]
        \ar[d, " G_2 ( \gamma_X ) "]
        \ar[rd, " ( \delta * \gamma )_X "]
        &
        G_3 ( F_2 ( X ) ) 
        \ar[d, " G_3 ( \gamma_X ) "]
        \\
        G_1 ( F_3 ( X ) ) 
        \ar[r, " \beta_{ F_3 ( X ) }"]
        &
        G_2 ( F_3 ( X ) )
        \ar[r , " \delta_{F_2 ( X ) }"]
        &
        G_3 ( F_3 ( X ) ) 
    \end{tikzcd}
    \]
    Now the long diagonal of the diagram corresponds to $ ( \delta * \gamma ) \circ ( \beta * \alpha ) $ and the outer large square to $ ( \delta \circ \beta ) * ( \gamma \circ \alpha )$.
\end{proof}

\begin{defi}
    The Godement products bellow are called whickerings
    \todo{missing}
\end{defi}

\begin{construction}
    Given a cospan of groupoids 
    \[
    \begin{tikzcd}
        &
        \mathcal{B}
        \ar[d, " G "]
        \\
        \mathcal{ A }
        \ar[r, " F "]
        &
        \mathcal{ C }
    \end{tikzcd}
    \]
    its 2-pullback is the diagonal of groupoids.
    \[
    \begin{tikzcd}
        \mathcal{ A } \times_{ \mathcal{ C } } \mathcal{ B } 
        \ar[r, " \pi_{ \mathcal{ B }} "]
        \ar[d, " \pi_{ \mathcal{ A }} "]
        &
        \mathcal{ B }
        \ar[ d , " G " ]
        \\
        \mathcal{ A }
        \ar[r, " F " ]
        &
        \mathcal{ C }
    \end{tikzcd}
    \]
    The objects are given as $ \Ob ( \mathcal{ A } \times_\mathcal{ C } \mathcal{ B } ) = ( a \in \mathcal{ A } , b \in \mathcal{ B }, \varphi \colon F ( a ) \isomorphism G ( b ) $ in $ \mathcal{ C } )$ and morphisms are given by tuples of morphisms $ ( u , v ) \colon ( a , b, \varphi ) \to ( a' , b' , \varphi' ) $, where $ u \colon a \to a' $ and $ v \colon b \to b '$ are morphisms in the respective groupoids, such that the following square commutes
    \[
    \begin{tikzcd}
        F ( a ) 
        \ar[r, "\varphi" , " \sim "']
        \ar[d, "F( a )"']
        &
        G ( b )
        \ar[d, " G ( v )"]
        \\
        F ( a' )
        \ar[r, "\varphi", "\sim"']
        &
        G(b')
    \end{tikzcd}
    \]
\end{construction}