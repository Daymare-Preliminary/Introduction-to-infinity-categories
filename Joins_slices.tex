3.7.2025

\section{Joins \& slices of $\infty$-categories}

The reference for this section is \cite[3.4]{Cisinski_2019}.

Let $ F \colon A \to \mathcal{ C } $ be a functor of 1-categories.

\underline{Recall}:
\newline
A limit of $ F $ is a final object in $ \mathcal{ C } / F $ the 1-category of cones over $ F $.
\newline
\underline{Aim}:
\newline
Extend the above to $ \infty $-categories.
\newline
Today:
The $ \infty $-categorical analogue of $ \mathcal{ C } / F $.
Let us begin by understanding $ \mathcal{ C } / X , X \in \mathcal{ C } $
\begin{itemize}
	\item 
	$ \Ob ( \mathcal{ C } / F ) $ is given by tuples $ ( C \in \mathcal{ C } , f \colon C \to X ) $,
	
	\item 
	Let $ ( C' , f' ) $ be another tuple, then a morphism is given by a morphism $ g \colon C \to C' )$ such that $ f = f' \circ g $.
\end{itemize}

So what does $ \Hom_{ \Cat } ( - , \mathcal{ C } / X ) = ? $ look like ? 
Let $ F \colon \mathcal{ D } \to \mathcal{ C } / X $ be a functor.
So for any morphism $ \varphi \colon d \to d' $ we obtain a commutative square
\[
\begin{tikzcd}
	d \in \mathcal{ D }
	\ar[r, mapsto]
	\ar[d, "\varphi"]
	& 
	F ( d ) = ( F ( d ) 
	\ar[d, "F ( \varphi )"' , shift right=5]
	\ar[d, "F ( \varphi )" , shift left=5]
	\ar[r, "f_d"]
	&
	X )
	\ar[d, equal]
	\\
	d' \in \mathcal{ D } 
	\ar[r , mapsto ]
	&
	F ( d' ) = ( F ( d' ) 
	\ar[r , " f_{ d' } "] 
	&
	X )
\end{tikzcd}
\]

Consider now the category $ \mathcal{ D } * \mathds{ 1 } $ called the join where $ \mathds{ 1 } = \{ \infty \} $.
\begin{enumerate}
	\item 
	Let $ \Ob ( \mathcal{ D } * \mathds{ 1 } ) = \Ob ( \mathcal{ D } ) \coprod \Ob ( \mathds{ 1 } ) = \Ob ( \mathcal{ D } ) \coprod \{ \infty \},$
	
	\item 
	the morphisms are given as 
	\begin{align*}
		\Hom_{ \mathcal{ D } * \mathds{ 1 } } ( d , d' ) 
		&=
		\Hom_{ \mathcal{ D } } ( d , d' )
		\\
		\Hom_{ \mathcal{ D } * \mathds{ 1 } } ( d , \infty )
		&=
		\{ u_{ \infty ,d } \colon d \to \infty \}
		\\
		\Hom_{ \mathcal{ D } * \mathds{ 1 } } ( \infty , d' )
		&=
		\emptyset
		\\
		\Hom_{ \mathcal{ D } * \mathds{ 1 } } ( \infty , \infty )
		&=
		\{ \id_\infty \}.
	\end{align*}
\end{enumerate}

From this we obtain the following commutative diagram
\[
\begin{tikzcd}
	\mathcal{ D } * \mathds{ 1 }
	\ar[rr, " \overline{F} "]
	&&
	\mathcal{ C }
	\\
	&
	\mathds{1}
	\ar[ru, "X"]
	\ar[lu, "\can"]
\end{tikzcd}
\]
where $ \overline{ F } ( d ) = \pi_\mathcal{C} \circ F ( d ) $ and $ \overline{ F } ( \infty ) = X $. 
From this we obtain
\[
\begin{tikzcd}
	d
	\ar[dd, "\varphi"']
	\ar[rd, "u_{ \infty , \overline{f} }"]
	\\
	&
	\infty
	\\
	d'
	\ar[ru, "u_{ d', \infty }"']
\end{tikzcd}
\xrightarrow{ \overline{ F } }
\begin{tikzcd}
	\overline{F} ( d )
	\ar[dd, " F ( \varphi ) "']
	\ar[rd, " f_d "]
	\\
	&
	\overline{ F } ( \infty ) = X 
	\\
	\overline{ F } ( d' )
	\ar[ru, " f_{ d' } "']
\end{tikzcd}
\]

Furthermore there is an adjunction isomorphism

\begin{align*}
	\Hom_{ \Cat } ( \mathcal{ D } , \mathcal{ C } / X )
	&\isomorphism
	\Hom_{ \Cat_{ \mathds{ 1 } / } } ( \mathds{ 1 } \xrightarrow{ \can } \mathcal{ D } * \mathds{ 1 } , \mathds{ 1 } \xrightarrow{ X } \mathcal{ C } )
	\\
	F
	&\mapsto
	\begin{tikzcd}[ampersand replacement=\&]
		\mathcal{ D } * \mathds{ 1 }
		\ar[rr, " \overline{F} "]
		\&\&
		\mathcal{ C }
		\\
		\&
		\mathds{1}
		\ar[ru, "X"']
		\ar[lu, "\can"]
	\end{tikzcd}
\end{align*}

written compactly
\[
\begin{tikzcd}
	-* \mathds{1} \colon \Cat 
	\ar[r, shift left]
	&
	\Cat_{ \mathds{1}/ } \colon R 
	\ar[l, shift left, "\adj"]
\end{tikzcd}
\]
where $ R ( \mathds{ 1 } ) \xrightarrow{ X } \mathcal{ C } ) \coloneqq \mathcal{ C } / X $.

\underline{More generally}: 
Consider $ \mathcal{ D } * \mathcal{ A } $ the category with 
\begin{itemize}
	\item 
	$ \Ob ( \mathcal{ D } * \mathcal{ A } ) = \Ob ( \mathcal{ D } ) \coprod \Ob ( \mathcal{ A } )$
	
	\item 
	morphisms 
	\[
		\Hom_{ \mathcal{ D } * \mathcal{ A } } ( x , y ) 
		\coloneqq
		\begin{cases}
			\Hom_{ \mathcal{ D } } ( x , y ) & x , y \in \mathcal{ D } 
			\\
			\Hom_{ \mathcal{ A } } ( x , y ) & x , y \in \mathcal{ A } 
			\\
			\{ u_{ y , x } \} & x \in \mathcal{ D } , y \in \mathcal{ A }
			\\
			\emptyset & x \in \mathcal{ A } , y \in \mathcal{ D }
		\end{cases}
	\]
\end{itemize}

Consider now a functor $ \phi \colon \mathcal{ D } \to \mathcal{ C } / F $ where $ F \colon \mathcal{ A } \to \mathcal{ C } $.
Then we have:
\[
\begin{tikzcd}
	d \in \mathcal{ D } 
	\ar[r, mapsto , "\phi"]
	\ar[d, "\varphi"]
	&
	\phi ( d )
	= ( \overline{ \phi } ( d ) 
	\ar[d," \phi ( \varphi ) "',shift right = 4]
	\ar[d, "\overline{ \phi } ( \varphi ) ", shift left = 4]
	\ar[r, "f_{ a , d }" ]
	& 
	F ( a ) )_{ a \in \mathcal{ A } }
	\ar[d, equal]
	\\
	d' \in \mathcal{ D } 
	\ar[r, mapsto]
	&
	\phi ( d' ) = ( \overline{ \phi } ( d' ) )
	\ar[r, " f_{ a , d' } "]
	&
	F ( a ) )_{ a in \mathcal{ A } }
\end{tikzcd}
\]
Analogously to the previous case we obtain a commutative triangle
\[
\begin{tikzcd}
	\mathcal{ D } * \mathcal{ A } 
	\ar[rr, " \overline{ \phi } "]
	&&
	\mathcal{ C }
	\\
	&
	\mathcal{ A }
	\ar[ru," F "']
	\ar[lu, "\can_{ \mathcal{ A } }" ]
\end{tikzcd}
\]
where $ \overline{ \phi } ( d ) = \pi_{ \mathcal{ C } } \circ \phi ( d ) , \overline{ \phi } ( a ) = F ( a ) , \overline{ \phi } ( u_{ a , d }  ) = f_{ a , d' }, \overline{\phi} ( \varphi \colon d \to d' ) = \pi_{ \mathcal{ C } } \circ \phi ( \varphi ) $ and $ \overline{ \phi } ( u \colon a \to a') = F ( a ) $.

\[
\begin{tikzcd}
	d
	\ar[rd, "u_{ a ,d } "]
	\ar[dd, "\varphi"']
	\\
	&
	a
	\\
	d'
	\ar[ru, "u_{ a , d' }"' ]
\end{tikzcd}
\xrightarrow{\overline{\phi}}
\begin{tikzcd}
	\overline{\phi} ( d )
	\ar[rd, "f_{ a ,d } "]
	\ar[dd, "\overline{ \phi } ( \varphi )"]
	\\
	&
	F ( a )
	\\
	\overline{\phi} ( d' )
	\ar[ru, "f_{ a , d' }"' ]
\end{tikzcd}
\] 

Again we obtain an adjunction

\[
	\Hom_{ \Cat } ( \mathcal{ D } , \mathcal{ C } / F )
	\isomorphism_{ \can }
	\Hom_{ \Cat_{ \mathcal{ A } / } } ( \mathcal{ A } \xrightarrow{ \can_{ \mathcal{ A } } } \mathcal{ D } * \mathcal{ A } , \mathcal{ A } \xrightarrow{ F } \mathcal{ C } )
\]

given compactly as 

\[
\begin{tikzcd}
	- * \mathcal{ A } \colon \Cat
	\ar[r, shift left]
	&
	\Cat_{ \mathcal{ A } / } \colon R 
	\ar[ l , shift left, "\adj" ]
\end{tikzcd}
\]
where $ R ( \mathcal{ A } \xrightarrow{ F } \mathcal{ C } ) = \mathcal{ C } / F $. 

This discussion suggests the following:
Construct a bifunctor 
\[
	- * - \colon \SetD \times \SetD \to \SetD	
\]
that is close to preserving small colimits in each variable seperately.
We want $ N ( \mathcal{ C } * \mathcal{ D } ) \cong N ( \mathcal{ C } ) * N ( \mathcal{ D } ), \mathcal{ C } , \mathcal{ D } \in \Cat $.
In particular : $ \delta^m * \Delta^n \cong N ( [ m ] * [ n ] ) \cong N ( [ m + n + 1 ] ) \cong \Delta^{ m + n + 1 } $, since
\[
\begin{tikzcd}
	\left[ m \right] * [ n ] = 0
	\ar[d, " \cong"]
	\ar[r]
	\ar[rrrrd]
	&
	1
	\ar[r]
	\ar[rrrd]
	&
	2
	\ar[r]
	\ar[rrd]
	&
	\dotsc
	\ar[r]
	\ar[rd]
	&
	m
	\ar[d]
	\ar[rd]
	\ar[rrd]
	\ar[rrrrd]
	\\
	\left[ m + n + 1 \right]
	&&&&
	0
	\ar[r]
	&
	1
	\ar[r]
	&
	2
	\ar[r]
	&
	\dotsc 
	\ar[r]
	&
	n
\end{tikzcd}
\]

\begin{defi}
	The \textbf{augmented simplex category} $ \Delta_{ \aug } = \Delta_+ $ has 
	\begin{itemize}
		\item 
		objects :$ [ - 1 ] = \emptyset , [ n ] = \{ 0 < 1 < \dotsc < n \}, n \geq 0 $
		
		\item 
		morphisms: monotone maps $ ( \phi \in \Delta_{ \aug } : \text{initial} ) $.
		
	\end{itemize}
	Then $ \Set_{ \Delta_ { \aug } } \coloneqq \Fun ( \Delta^{ \op }_{ \aug } , \Set ) $ is the category of augmented simplicial sets.
\end{defi}

\begin{rmk}
	An augmented simplicial set is equivalent to the data of a simplicial set $ X $ together with a morphism $ X \xrightarrow{\lambda} N ( E ) , E \in \Set $
	\[
	\begin{tikzcd}
		X : \dotsc 
		\arrow[r, altstackar=7]
		&
		X_2 
		\ar[d]
		\arrow[r, altstackar=5]
		&
		X_1
		\ar[d]
		\arrow[r, altstackar=3]
		&
		X_0
		\ar[d]
		\\
		E : \dotsc
		\arrow[r, altstackar=7]
		&
		E
		\arrow[r, altstackar=5]
		&
		E
		\arrow[r, altstackar=3]
		&
		E
	\end{tikzcd}
	\]
	Take the canonical inclusion
	\[
		i \colon \Delta \hookrightarrow \Delta-{ \aug } \xrightarrow{ X } \Set^{ \op }
	\]
	by Kan's extension theorem we obtain an adjunction
	\[
	\begin{tikzcd}
		\Set_{ \Delta_{ \aug } } 
		\ar[ r , " i^* "]
		&
		\Set_{ \Delta } 
		\ar[l , bend right, "i_!"']
		\ar[l , bend left , "i_*"]
	\end{tikzcd}
	\]
\end{rmk}

\todo{ missing stuff, Philipp fragen oder morgen denken }

\begin{lem}	
	The augmented simplex category $ \Delta_{ \aug } $ admits a monoidal structure.
	\begin{align*}
		\Delta_{ \aug } \times \Delta_{ \aug } 
		&\xrightarrow{ - * - } 
		\Delta_{ \aug }
		\\
		( [ n ] , [ m ] ) 
		&\mapsto
		[ n + m + 1 ]
	\end{align*}
	where the composition is called \textbf{ordinal sum}.
\end{lem}

\begin{construction}{Day Convolution}
	\newline
	The category $ \Fun ( \Delta^{ \op }_{ \aug } , \Set ) $ inherits a monoidal structure as follows, take
	\[
		-*- \colon \Delta_{ \aug } \times \Delta_{ \aug } 
		\to
		\Delta_{ \aug } 
		\xrightarrow{ \text{Yoneda} }
		\Set_{ \Delta_{\aug} } 
	\]
	extending by colimits we obtain
	\[
	\begin{tikzcd}
		\Fun ( \Delta_{ \aug }^{ \op } \times \Delta_{ \aug }^{ \op }, \Set )
		\ar[r]
		\ar[d, equal]
		&
		\Set_{ \Delta_{ \aug } } 
		\\
		\Set_{ \Delta_{ \aug } \times \Delta_{ \aug } }
		\\
		\Set_{ \Delta_{ \aug } \times \Delta_{ \aug } } 
		\ar[ u , " - \boxtimes - " ]
		\ar[ruu, " - * - "']
	\end{tikzcd}
	\]
	Here 
	\[
		X \boxtimes Y : ( [ k ] , [ l ] ) \mapsto X_k \times Y_l. 
	\]
	One can show that $ ( X * Y )_n = \coprod_{ i + j + 1 = n } X_i \times Y_j $ and that $ - * - \colon \Set_{ \Delta_{ \aug } } \times \Set_{ \Delta_{ \aug } } \to \Set_{ \Delta_{ \aug } } $
	preserves colimits in each variable seperately.
	Moreover 
	\[
		y [ n ] * y [ m ] 
		= 
		y [ n + m + 1 ]
		\cong
		y ( [ n ] * [ m ] )
	\]
	In order to define $ - * - \colon \Set_\Delta \times \Set_\Delta \to \Set $ we let 
	$ X *_{ \Delta } Y \coloneqq i^* (i_* ( X ) *_{ \Delta_{ \aug } } i_* ( Y ) ) $.
\end{construction}

One can show that $ - * Y \colon \Set_\Delta \to ( \Set_\Delta )_{ Y / } $ preserves small colimits 

\todo{ fill in missing, do not understand right now} 

Lecture 8.07.2025

Let $ F \colon A \to \mathcal{ C } $ and $ \mathcal{ C } $ be an $ \infty $-category take $ \mathcal{ C } / F $ and consider 
\[
	\Hom_{ \Set_\Delta } ( K , \mathcal{ C } / F ) 
	\isomorphism 
	\Hom_{ A / \SetD } ( A \to K * A , A \xrightarrow{ F } \mathcal{ C } )
\]
We can consider what happens to the n-simplices under this isomorphism
\[
	\Hom_{ \Set_\Delta } ( \Delta^n , \mathcal{ C } / F )
	\cong
	( \mathcal{ C } / F )_n
\]
we obtain a correspondence 
\[
	\sigma \colon \Delta^n \to \mathcal{ C } / F 
	\leftrightarrow
	\begin{tikzcd}
		\Delta^n * A 
		\ar[rr, " \overline{ F } "]
		&&
		\mathcal{ C }
		\\
		& 
		A
		\ar[lu]
		\ar[ru," F "']
	\end{tikzcd}
\]

\todo{ pictures } 

Consider the projection $ \mathcal{ C } / F \xrightarrow{ p } \mathcal{ C } $ and 
\todo{ too tired to think about what happens here right now}

\begin{prop}
	The following types of lifting problems correspond to each other $ ( i \colon K \to L , p \colon X \to Y ) $
	\[
	\begin{tikzcd}
		A 
		\ar[r]
		\ar[d]
		&
		\underline{\Hom}( L , X )
		\ar[d, "\nu = \nu_{ i , p }" ]
		\\
		B 
		\ar[ru, dashed]
		\ar[r]
		&
		\underline{\Hom} ( K , X ) \times_{ \underline{ \Hom ( K , Y ) } } \underline{\Hom} ( L , Y ) 
	\end{tikzcd}
	\quad
	\begin{tikzcd}
		( A \times L ) \coprod_{ ( A \times K ) } ( B \times K )
		\ar[d]
		\ar[r]
		&
		X
		\ar[d, "p"]
		\\
		B \times L 
		\ar[ru, dashed]
		\ar[r]
		&
		Y
	\end{tikzcd}
	\]
\end{prop}

This has as a consequence that the following statements are equivalent for $ p \colon X \to Y $:
\begin{enumerate}
	\item 
	$ p \colon X \to Y $ is an inner fibration,
	
	\item 
	for all $ i \colon L \hookrightarrow K $ inner anodyne $ u_{ i , p } $ is a trivial fibration,
	
	\item 
	$ \twoHom ( \Delta^2 , X ) \to \twoHom ( \Lambda_1^2 , X ) \times_{ \twoHom ( \Lambda_1^2 , Y ) } \twoHom ( \Delta^2 , Y ) $ is a trivial fibration.
	
	\item 
	For all $ i \colon L \hookrightarrow K $ monomorphisms $ \nu_{ i , p } $ is an inner fibration, taking $ Y = \Delta^0 , K = \emptyset , X \infty \text{-category} \implies \twoHom ( L , X )$ is an $ \infty $-category. 
\end{enumerate}

Moreover, $ X $ is an infinity category is equivalent to $ \twoHom ( \Delta^2 , X ) \to \twoHom ( \Lambda^2 , X ) $ is a trivial fibration.

\todo{ check this }

\begin{defi}
	A map $ p \colon X \to Y $ in $ \Set_\Delta$ is a \textbf{right fibration} if
	$ \forall n \geq 1 , \forall 0 \leq k < n $ there exists a lift
	\[
	\begin{tikzcd}
		\Lambda_k^n 
		\ar[r, " \forall \sigma "]
		\ar[d]
		&
		X
		\ar[d, "p"]
		\\
		\Delta^n
		\ar[ru, dashed, "\exists"]
		\ar[r, "\forall \tau"]
		&
		Y
	\end{tikzcd}
	\]
\end{defi}

\begin{prop}
	Let $ p \colon X \to Y $ be a right fibration and $ i \colon L \hookrightarrow K $ a monomorphism, then $ \nu \colon \twoHom ( K , X ) \to \twoHom ( L , X ) \times_{ \twoHom ( L , Y ) } \twoHom( K , Y ) $ is a right fibration.
\end{prop}

Informally speaking right fibrations correspond to presheaves of $ \infty $-groupoids.

\begin{prop}
	Let $ i \colon A \hookrightarrow B , j \colon S \hookrightarrow T $ be monomorphisms and $ p \colon X \to Y $ a morphism.
	The following lifitng problems correspond to each other:
	\[
	\begin{tikzcd}
		( B * S ) \coprod_{ ( A * S ) } ( A * T )
		\ar[r]
		\ar[d]
		&
		X
		\ar[d, "p "]
		\\
		B * T 
		\ar[r]
		&
		Y
	\end{tikzcd}
	\quad 
	\begin{tikzcd}
		A 
		\ar[r]
		\ar[d]
		&
		X / T
		\ar[d]
		\\
		B
		\ar[r]
		&
		X / S *_{ Y / S } Y / T
	\end{tikzcd}
	\]
	coming from squares 
	\[
	\begin{tikzcd}
		A * S 
		\ar[r, "A * j"]
		\ar[d, "i * S"]
		&
		A*T 
		\ar[d, "i * T"]
		\\
		B * S 
		\ar[r, " B * j "]
		&
		B*T
	\end{tikzcd}
	\quad 
	\begin{tikzcd}
		X / T 
		\ar[ r , " X / j " ]
		\ar[ d ]
		&
		X / S
		\ar[ d ]
		\\
		Y / T
		\ar[r]
		&
		Y / S
	\end{tikzcd}
	\]
	coming from squares
	\[
	\begin{tikzcd}
		A * S 
		\ar[r, " A * j "]
		\ar[d, " i * S "]
		&
		A*T
		\ar[d, " i * T "]
		\\
		B * S \ar[r, " B * j " ]
		\ar[r, " B * j "]
		&
		B * T
	\end{tikzcd}
	\quad
	\begin{tikzcd}
		X / T 
		\ar[ r , " X / j "]
		\ar[d]
		&
		X / S
		\ar[d]
		\\
		Y / T 
		\ar[r]
		&
		Y / S
	\end{tikzcd}
	\]
\end{prop}

\begin{prop}
	Let $ p \colon X \to Y $ be an inner fibration and $ S \hookrightarrow T \xrightarrow{ t } X $ a composition, then $ \nu \colon X / T \to ( X / S ) \times_{ Y / S } ( Y / T ) $ is a right fibration and if $ Y $ is an $ \infty $-category, then $ u , v $ are $ \infty $-categories.
	\todo{stuff missing}
\end{prop}

\begin{thm}{Joyal}
	Let $ X $ be an $ \infty $-category $ X $ a Kan complex if and only if $ X $ is an $ \infty $-groupoid if and only if $ \tau X $ is a groupoid.
\end{thm}

\begin{prop}
	Let $ p \colon X \to Y $ be a right fibration, if $ Y $ is a Kan complex, then $ p $ is a Kan fibration. 
\end{prop}

\begin{cor}
	Let $ p \colon  X \to Y $ be a right fibration, then its fibres are Kan complexes.
\end{cor}

\begin{defi}
	An object $ X \in \mathcal{ C } $ where $ \mathcal{ C } $ is an $ \infty $-category is \textbf{(strongly) final} if the right fibration $ \mathcal{ C } / X \to \mathcal{ C } $ has contractible fibres, meaning it is a trivial fibration, put differently the the fibres are given by the following pullbacks.
	\[
	\begin{tikzcd}
		\Delta^0 \cong \Hom_{ \mathcal{ C } }^R ( Y , X )
		\ar[r]
		\ar[d]
		&
		\mathcal{ C } / X 
		\ar[d]
		\\
		\Delta^0
		\ar[r, "\forall Y "]
		&
		\mathcal{ C }
	\end{tikzcd}
	\]
\end{defi}

\begin{defi}
	Let $ F \colon A \to \mathcal{ C } $ be an $ \infty $-category $ A $ a limit of $ F $ is a final object of $ \mathcal{ C } / F $.
\end{defi}

\subsection{Exercises}

\begin{Exercise}
	Consider the augmented simplex category $ \Delta_{ \aug } $ which extend $ \Delta $ by the initial object $ \left[ - 1 \right] \coloneqq \emptyset $ as well as the inclusion $ \emptyset \subseteq \left[ n \right] $.
	We denote by $ \iota \colon \Delta \to \widehat{\Delta}_{ \aug } $ the canonical inclusion and denote by $ \widehat{ \Delta }_{ \aug } \coloneqq \Fun ( \Delta_{ \aug } , \Set ) $ the category of augmented simplicial sets.
	
	\begin{enumerate}[label=(\alph*)]
		\item 
		Show that $ \iota^* \colon \widehat{\Delta}_{ \aug }
		\to \SetD $ admits both a left and a right adjoint , $ \iota_! \adj \iota^* \adj \iota_* $ , and describe them as objects. 
	\end{enumerate}
	
	We equip $ \Delta_{ \aug } $  with a monoidal structure $ \left[ m \right] \star \left[ n \right] \coloneqq \left[ m + 1 + n \right] $ with unit object $ \left[ - 1 \right] $.
	Following Exercise 1.3 we can extend the functor $ \star \colon \Delta_{ \aug } \times \Delta_{ \aug } \to \Delta_{ \aug } \to \widehat{ \Delta }_{ \aug } $ to a monoidal product on the presheaf category $ \widehat{ \Delta }_{ \aug } $ using the canonical map.
	\begin{align*}
		\can \colon \widehat{ \Delta }_{ \aug } \times \widehat{ \Delta }_{ \aug } 
		&\to 
		\Fun ( \Delta^{ \op }_{ \aug } \times \Delta^{ \ op }_{ \aug } , \Set ) 
		\\
		( X , Y )
		&\mapsto
		( X \boxtimes Y \colon ( m , n ) \mapsto X ( m ) \times Y ( n ) )
	\end{align*}
	(This induced monoidal structure on the presheaf category is also known as Day convolution.)
	
	\begin{enumerate}[label=(\alph*),resume]
		\item 
		Show that for any augmented simplicial set $ X \in \widehat{ \Delta }_{ \aug } $ both functors 
		\[
		- \star X , X \star - \colon 
		\widehat{ \Delta }_{ \aug }
		\to 
		\widehat{ \Delta }_{ \aug }
		\]
		are colimit preserving.
		
		\item 
		Show that for two augmented simplicial sets $ X , Y \in \widehat{ \Delta }_{ \aug }$, we have $ ( X \star Y )_n 0 
		\coprod_{ i + 1 + j = n } X_i \times X_j $ for $ - 1 \leq n \in \mathbb{ Z } $. 
		Furthermore, explain the difference between $ X \star Y $ and $ Y \star X $.
		\newline
		(Hint: Observe that $ \Hom ( \left[	n \right] , \left[	p + 1 + q \right] ) = \coprod_{ i + j + 1 = n } \Hom ( \left[ i \right] , \left[ p \right ] ) \times \Hom ( \left[ j \right] , \left[ q \right] ) $ in $ \Delta_{ \aug } .$)
	\end{enumerate}	
	
	We define a monoidal product $ \star \colon \SetD \times \SetD \to \SetD $ by $ X \star Y  \coloneqq \iota^* ( \iota_* X \star \iota_* Y ) $ and call $ X \star Y $ the join of $ X $ and $ Y $.
	Notice that $ X \star Y $ canonically contains $ X $ and $ Y $ as simplicial subsets (set $ j = - 1 $ respective $ i = - 1 $ in (c)) so that we get functors 
	\[
	X \star - \colon \SetD \to X / \SetD 
	\qquad 
	- \star Y \colon \SetD \to Y / \SetD
	\]  
	where $ X / \SetD $ is the coslice category, i.e. the category whose objects are morphisms $ X \to Z $ and morphisms are commuting triangles. 
	Recall that for a functor $ D \colon I \to X / \SetD $ its colimit is given by the colimit of $ D_X \colon \emptyset \trianglelefteq I \to \SetD $ together with the induced map $ X = D_X ( \emptyset ) \to \colim_{ \emptyset \trianglelefteq I } D_X.$
	Here
	$ \emptyset \trianglelefteq I $ is the category that augments $ I $ by an initial element $ \emptyset $ and $ D_X ( \emptyset \to i ) \coloneqq D ( i ) $.
	
	\begin{enumerate}[label=(\alph*),resume]
		\item 
		Utilize Exercise 1.3 to show that there exist adjoints $ X \star - \adj X \setminus ? $ and $ - \star X \adj ? / X $ called left and right slice.
		
		\item 
		Follow a similar strategy as the one outlined in Exercise 6.4 to show that a morphism of simplicial sets $ f \colon X \to Y $ induces morphisms of adjunctions
		\begin{align*}
			( X \star - \adj X \setminus ? ) \to ( Y \star - \adj Y \setminus ? )
			\qquad
			\text{ and }
			\qquad
			( - \star X \adj ? / X )
			\to 
			( - \star Y \adj ? / Y )
		\end{align*}
		in the sense of Exercise 8.3
	\end{enumerate}
	Setting $ X = \Delta^0 $, the above construction can be used to define the $ \infty $-categorical version of the slice and coslice category.
\end{Exercise}

\begin{Exercise}
	We aim to show that for an inner fibration $ p \colon X \to Y $ betweeen $ \infty $-categories, for any monomorphism $ A \hookrightarrow B $ any lifting problem
	\[
	\begin{tikzcd}
		\{ 0 \} \star B \cup \Delta^1 \star A
		\ar[d, hook]
		\ar[r, "f"]
		& 
		X
		\ar[d, "p"]
		\\
		\Delta^1 \star B 
		\ar[r, " g"' ]
		&
		Y
	\end{tikzcd}
	\]
	admits a solution if the edge $ \alpha \colon \Delta^1 \hookrightarrow \Delta^1 \star T 
	\xrightarrow{ g } $ is invertible.
	So we fix an inner fibration $ p \colon X \to Y $ and a monomorphism $ A \hookrightarrow B $.
	Here we used that by the description of the join in Exercise 10.2, the join of two monomorphisms is a monomorphism so that the notation
	\[
	L \star A \cup K \star B 
	\coloneqq 
	L \star A 
	\coprod_{ A \star K } 
	K \star B 
	\hookrightarrow 
	L \star B
	\]
	makes sense for the simplicial sets $ K \subseteq L $.
	\begin{enumerate}[label=(\alph*)]
		\item 
		Utilize Exercise 8.3 to deduce from Exercise 10.1 that it suffices to show that the induced morphism 
		\[
		X / B 
		\to 
		X / A 
		\times_{ Y / A }
		Y / B 
		\]
		is a right fibration between $ \infty $-categories with $ \alpha' \colon \Delta^1 \to X / A \times_{ Y / A } Y / B $ an invertible edge.
		
		\item 
		Show that with the help of Exercise 8.3
		that 
		\[
		X / B 
		\to
		X / A \times_{ Y / A } Y / B  
		\]
		is a right fibration if 
		\[
		\Delta^n \setminus X 
		\to
		\Lambda_k^n \setminus X \times_{ Lambda_k^n \setminus Y } \times \Delta^n \setminus Y
		\]
		is a trivial fibration for all $ 0 < k \leq n $.
		
		\item 
		Show one of the following identities for subobjects of $ \Delta^{ m + 1 + n } = \Delta^m \star \Delta^n $.
		\begin{align}
			\partial \Delta^m \star \Delta^n \cup \Delta^m \star \partial \Delta^n
			&=
			\partial \Delta^{ m + 1 + n }
			\\
			\Lambda_k^m \star \Delta^n \cup \Delta^m \star \partial \Delta^n
			&=
			\Lambda_k^{ m + 1 + n }
			\\
			\partial \Delta^m \star \Delta^n \cup \Delta^m \star \Lambda_k^n 
			&=
			\Lambda_{ m + 1 + k }^{ m + 1 + n }
		\end{align}
		
		\item 
		Deduce that $ \Delta^n \setminus X  \to \Lambda_k^n \setminus X \times_{ \Lambda_k^n \setminus Y } \Delta^n \setminus Y $ is a trivial fibration for all $ 0 < k \leq n $ and any inner fibration $ p \colon X \to Y $.
		
		\item 
		Prove Joyal's Theorem, i.e. that the initially posed lifting problem admits a solution.
		\newline
		(Hint: Consider the case where $ Y = \Delta^0 $ is final and $ A = \emptyset $ is initial to show that $ X / A \times_{ Y / A } Y / B $ is an $\infty$-category if $ Y $ is an $ \infty $-category. Then use the same diagrams to show that $ \alpha' $ is invertible.) 
	\end{enumerate} 
\end{Exercise}

\begin{Exercise}
	\begin{enumerate}[label=(\alph*)]
		\item 
		Let $ p \colon X \to Y $ be an inner fibration between $ \infty $-categories. Deduce from Exercise 10.3 that for $ n \geq 2 $ any lifting problem 
		\[
		\begin{tikzcd}
			\Lambda_0^n 
			\ar[r, "f"]
			\ar[d, hook]
			&
			X
			\ar[d, "p"]
			\\
			\Delta^n 
			\ar[r, "g"]&
			Y
		\end{tikzcd}
		\]
		where the edge $ \alpha \colon \Delta^1 \subseteq \Delta^n \xrightarrow{ g } $ is invertible, admits a solution, i.e. some $ h \colon \Delta^n \to X $ such that $ f = h \circ \iota $ and $ g = p \circ h $.
		
		\item 
		Conclude that if $ p \colon X \to Y $ is a right fibration and $ Y $ is a Kan complex, then $ p $ is a Kan fibration and $ X $ is a Kan complex.
		
		\item 
		Conclude that an $ \infty $-category $ X $ is an $ \infty $-groupoid in the sense of Exercise 10.1 if and only if $ X $ is a Kan complex.
	\end{enumerate}
\end{Exercise}

