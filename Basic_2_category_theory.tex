Lecture 17.04

\section{2-categories}

Reference for this section is \cite[ch. 2.2.1- 2.2.3]{kerodon}.

\begin{defi}
\label{strict_twocat_defi}
	A \textbf{strict 2-category} $ \mathcal{ C } $ 
	consists of:
	\begin{enumerate}
		\item 
		A class $ \Ob ( \mathcal{C} ) $ of objects of $ \mathcal{C} $
		
		\item 
		For all $ X , Y \in \Ob( \mathcal{ C } ) $ a category $ \Hom_{ \mathcal{ C } } ( X , Y )$ whose objects 
		$f \colon X \to Y $ are called $1$-morphisms and whose morphisms $ \alpha \colon f \Rightarrow g $ are called 2-morphisms, with a vertical composition of 2-morphisms, that 
		is associative, and unital.
		
		\item 
		For all $ X , Y , Z \in \Ob( \mathcal{C} ) $ a horizontal composition functor
		\[
		-\circ- \colon \Hom_{ \mathcal{ C } } ( Y , Z ) \times \Hom_{ \mathcal{ C } } ( X , Y ) \to \Hom_{ \mathcal{ C } } ( X , Z )
		\]
		that is compatible with the vertical composition, in the following way 
		\[
		\begin{tikzcd}
			X
			\arrow[bend left=50]{rr}[name=U',label=above:$f$]{}
			\arrow[bend right=50]{rr}[name=D',label=below:$f'$]{}
			&&
			Y
			\arrow[shorten <=10pt,shorten >=10pt,Rightarrow,to path={(U')-- node[label=right:$ \alpha $] {} (D')}]{} 
			\arrow[bend left=50]{rr}[name=U'',label=above:$g$]{}
			\arrow[bend right=50]{rr}[name=D'',label=below:$g'$]{}
			&&
			Z
			\arrow[shorten <=10pt,shorten >=10pt,Rightarrow,to path={(U'')-- node[label=right:$\beta$] {} (D'')}]{} 
		\end{tikzcd}
		\mapsto
		\begin{tikzcd}[column sep=1.4cm]
			X
			\arrow[bend left=50]{rr}[name=U',label=above:$g \circ f$]{}
			\arrow[bend right=50]{rr}[name=D',label=below:$g' \circ f'$]{}
			&&
			Z
			\arrow[shorten <=10pt,shorten >=10pt,Rightarrow,to path={(U')-- node[label=right:$ \beta * \alpha $] {} (D')}]{} 
		\end{tikzcd} 	
		\]
		
		\item 
		Functoriality of horizontal composition:
		\[
		\begin{tikzcd}
			X
			\arrow[bend left=50]{rr}[name=U',label=above:$f$]{}
			\arrow[bend right=50]{rr}[name=D',label=below:$f$]{}
			&&
			Y
			\arrow[shorten <=10pt,shorten >=10pt,Rightarrow,to path={(U')-- node[label=right:$ \id_f $] {} (D')}]{} 
			\arrow[bend left=50]{rr}[name=U'',label=above:$g$]{}
			\arrow[bend right=50]{rr}[name=D'',label=below:$g$]{}
			&&
			Z
			\arrow[shorten <=10pt,shorten >=10pt,Rightarrow,to path={(U'')-- node[label=right:$ \id_g $] {} (D'')}]{} 
		\end{tikzcd}
		\mapsto
			\begin{tikzcd}[column sep=1.4cm]
			X
			\arrow[bend left=50]{rr}[name=U',label=above:$g \circ f$]{}
			\arrow[bend right=50]{rr}[name=D',label=below:$g \circ f$]{}
			&&
			Z
			\arrow[shorten <=10pt,shorten >=10pt,Rightarrow,to path={(U')-- node[label=right:$ \id_{ g \circ f } $] {} (D')}]{} 
		\end{tikzcd} 
		\]
		
		\item 
		There is a composition
		\[
		\begin{tikzcd}[column sep=huge]	
			X
			\arrow[bend left=90]{r}[name=U,label=above:$f_1$]{}
			\arrow[bend right=90]{r}[name=D,label=below:$f_3$]{}
			\arrow{r}[name=M]{}
			\arrow{r}[label=below:$f_2$, pos=0.2]{}
			&
			Y
			\arrow[shorten <=5pt,shorten >=5pt,Rightarrow,to path={(U) -- node[label=right:$\alpha$] {} (M)}]{}
			\arrow[shorten <=5pt,shorten >=5pt,Rightarrow,to path={(M) -- node[label=right:$\gamma$] {} (D)}]{}
			\arrow[bend left=90]{r}[name=U',label=above:$g_1$]{}
			\arrow[bend right=90]{r}[name=D',label=below:$g_3$]{}
			\arrow{r}[name=M']{}
			\arrow{r}[label=below:$g_2$, pos=0.2]{}
			&
			Z
			\arrow[shorten <=5pt,shorten >=5pt,Rightarrow,to path={(U') -- node[label=right:$\beta$] {} (M')}]{}
			\arrow[shorten <=5pt,shorten >=5pt,Rightarrow,to path={(M') -- node[label=right:$\delta$] {} (D')}]{}    	
		\end{tikzcd}	
		\mapsto 
		\begin{tikzcd}
			X
			\arrow[bend left=50]{rr}[name=U',label=above:$g_1 \circ f_1$]{}
			\arrow[bend right=50]{rr}[name=D',label=below:$g_3 \circ f_3$]{}
			&&
			Z
			\arrow[shorten <=10pt,shorten >=10pt,Rightarrow,to path={(U')-- node[label=right:$ \eta $] {} (D')}]{}
		\end{tikzcd}	
		\]
		where $\eta $ is the Godement product $( \delta * \gamma ) \circ ( \beta * \alpha ) $.
	\end{enumerate}

The above data should satisfy the following axioms:
\begin{itemize}
	\item 
	(Unitality) 
	For all $ X \in \Ob ( \mathcal{ C } ) $ there exists $ \id_X \in \Ob ( \Hom_{ \mathcal{C } } ( X, X ))$ an identity 1-morphism such that for all $ Y \in \Ob( \mathcal{ C } ) $, the functor 
	\begin{align*}
		\Hom_{ \mathcal{ C } } ( X , Y ) \times \mathds{ 1 } &\to \Hom_{ \mathcal{ C } } ( X , Y ) \times \Hom_{ \mathcal{ C } } ( X , X ) \to \Hom_{ \mathcal{C } } ( X, Y )
		\\
		f &\mapsto (f, \id_X ) \mapsto f \circ \id_X
	\end{align*}
is equal to the identity functor $ \Hom_{ \mathcal{ C } } ( X , Y ) \to \Hom_{ \mathcal{ C }  } ( X, Y )$.
Similarly the functor
	\begin{align*}
		\mathds{ 1 } \times \Hom_{ \mathcal{ C } } ( Y , X ) &\to \Hom_{ \mathcal{ C } } ( X , X ) \times \Hom_{ \mathcal{ C } } ( Y , X ) \to \Hom_{ \mathcal{C } } ( Y, X )
		\\
		g &\mapsto \id_X \circ g = g
	\end{align*}
is equal to the identity functor $ \Hom_{ \mathcal{ C } } ( Y , X ) \to \Hom_{ \mathcal{ C }  }(  Y , X )$.

	\item 
	(Associativity) 
	For all $ W , X , Y , Z \in \Ob ( \mathcal{ C } ) $ the following square of functors commutes \underline{strictly}.
	\[
	\begin{tikzcd}[column sep= 3cm]
		\Hom_{ \mathcal{ C } } ( Y , Z ) \times \Hom_{ \mathcal{ C } } ( X , Y ) \times \Hom_{ \mathcal{ C } } ( W , X )
		\ar[r, " { \Hom_{ \mathcal{ C } } ( Y , Z ) \times ( - \circ - ) } "]
		\ar[d, " { ( - \circ - ) \times \Hom_{ \mathcal{ C } } ( W , X ) } "]
		&
		\Hom_{\mathcal{ C } } ( Y , Z ) \times \Hom_{ \mathcal{ C } } ( W , X )
		\ar[d, " { ( - \circ - ) } "]
		\\
		\Hom_{ \mathcal{ C } } ( X , Z ) \times \Hom_{\mathcal{ C } } ( W , X )
		\ar[r, " { ( - \circ -) } "]
		&
		\Hom_{ \mathcal{ C } }  ( W , Z )
	\end{tikzcd}
	\]
\end{itemize}
\end{defi}

\begin{defi}
\label{strict_moindal_cat_defi}
	A \textbf{strict monoidal category} is a strict 2-category with a single object, that is the following data:
	\begin{enumerate}
		\item 
		A strict 2-category $B \mathcal{ M }$ with $\Ob ( B \mathcal{ M } ) = \{ * \} $.
		
		\item 
		A category $ \mathcal{ M } \coloneqq \Hom_{ B \mathcal{ M }  } ( * , * ). $
		
		\item 
		A monoidal composition $ - \otimes - \colon \mathcal{ M } \times \mathcal{ M } \to \mathcal{ M } $, that fullfills 
		the following axioms.
		
		\begin{itemize}
			\item 
			(Unitality)
			There exists $ \mathds{ 1 }_{ \mathcal{ M } } \in \Ob( \mathcal{ M } ) = \Ob ( Hom_{ B \mathcal{M} } ( * , * ) )$ such that the functor 
			\begin{align*}
				\mathcal{ M } 
				& \to
				\mathcal{ M }
				\\
				M
				&\mapsto 
				M \otimes \mathds{ 1 }_{ \mathcal{ M } }
			\end{align*}
			is the identity, meaning there are equalities $ M \otimes  \mathds{ 1 }_{ \mathcal{ M } } = M = \mathds{ 1 }_{ \mathcal{ M } } \otimes M $.
			
			\item 
			(Associativity)
			The following square commutes
			\[
			\begin{tikzcd}[column sep = 2cm]
				\mathcal{ M } \times \mathcal{ M } \times  \mathcal{ M }
				\ar[r, " \id_{ \mathcal{ M } } \times  ( - \circ - ) "]
				\ar[d, " ( - \otimes - ) \times \id_{ \mathcal{ M } } "']
				&
				\mathcal{ M  } \times \mathcal{ M }
				\ar[d, "\otimes "]
				\\
				\mathcal{ M } \times \mathcal{ M }
				\ar[r, " - \otimes - " ]
				&
				\mathcal{ M } 
			\end{tikzcd}
			\]
			that is for all $ M_1 , M_2 , M_3 \in \mathcal{ M } $ 
			it holds that $ M_1 \otimes  ( M_2 \otimes M_3 ) = ( M_1 \otimes M_2 ) \otimes M_3 $.
		\end{itemize}
	\end{enumerate}		
\end{defi}

\begin{defi}
\label{strict_twofun_defi}
	Let $ \mathcal{ C } , \mathcal{ D } $  be strict 2-categories.
	A \textbf{strict 2-functor} $ F \colon \mathcal{ C } \to \mathcal{ D } $ consists of 
	\begin{enumerate}
		\item 
		A map $ F \colon \Ob ( \mathcal{ C } ) \to \Ob ( \mathcal{ D } )$  sending an object $ X $ to $ F ( X ) $.
		
		\item 
		For all $ X , Y \in \Ob ( \mathcal{ C } ) $ a functor $ F_{ X Y } \colon \Hom_{ \mathcal{ C } } ( X , Y ) \to \Hom_{ \mathcal{ D } } ( F X , F Y ) $ such that the following hold:
		\begin{itemize}
			\item 
			(Unitality)
			$ \forall X \in \Ob ( \mathcal{ C } ) , F ( \id_X ) = \id_{ F X } \in \Hom_{ \mathcal{ D } } ( F X, F X )$
				
			\item 
			(Composition)
			Let $ X, Y , Z \in \Ob ( \mathcal{ C } ) $
			\[
			\begin{tikzcd}
				\Hom_{ \mathcal{ C } } ( Y, Z ) \times \Hom_{ \mathcal{ C } }( X , Y )
				\ar[r, " - \circ^{ \mathcal{ C } } - "]
				\ar[d, " F_{ Y , Z } \times F_{ X , Y } "]
				&
				\Hom_{ \mathcal{ C } } ( X , Z )
				\ar[d, " F_{ X , Z } "]
				\\
				\twoHom_{ \mathcal{ D } } ( F Y , F Z ) \times \twoHom_{ \mathcal{ D } } ( F X , F Y )
				\ar[r, " - \circ^{ \mathcal{ D } } -"]
				&
				\twoHom_{ \mathcal{ D } }  ( F X , F Z )
			\end{tikzcd}
			\]
		\end{itemize}
	\end{enumerate}
	Let $ \mathcal{ C } $ be a strict 2-category then the opposite category $ \mathcal{ C }^{ \op } $ is also a strict 2-category.
\end{defi}

Lecture 22.04

\begin{rmk}
	Every ordinary 1-category $ \mathcal{ C } $  can be viewed as a strict 2-category as follows:
	\begin{itemize}
		\item 
		for all $ X , Y \in \Ob ( \mathcal{ C } ) , \Hom_{\mathcal{ C } } ( X , Y ) \coloneqq \Hom_{ \mathcal {C } }  ( X , Y ),$
		
		\item 
		horizontal composition = composition in $ \mathcal{ C } $,
		
		\item 
		for all $ X \in \Ob ( \mathcal{ C } )\label{key} $ it holds that $ \id_X \in \Hom_ { \mathcal{ C } } ( X , X ) = \Hom_{ \mathcal{ C } } ( X , X ) $ is the identity morphism in the original catgory $ \mathcal{ C }$,
			
		\item 
		conversely every strict 2-catgory $ \mathcal{ C } $ has an underlying ordinary category $ \mathcal{ C }_0 $ with $ \Ob ( \mathcal{ C }_0 ) \coloneqq \Ob ( \mathcal{ C } )$
		and $ \forall X , Y \in \Ob ( \mathcal{ C }_0 ) = \Ob ( \mathcal{ C } ), \Hom_{ \mathcal{ C }_0 } ( X , Y  ) \coloneqq \Ob ( \Hom_{ \mathcal{ C } }( X ,Y ) ).$
		The composition law in $ \mathcal{ C }_0 $ is horizontal composition of 1-morphisms in $ \mathcal{ C } $, this composition is associative since $ \mathcal{ C } $ is a strict 2-category.  
	\end{itemize}
\end{rmk}

\begin{defi}
\label{twocat_defi}
	A \textbf{2-category} (bicategory) $ \mathcal{ C }$ is given as follows:
	\begin{itemize}
		\item 
		a class of objects of $ \mathcal{ C } $ denoted $ \Ob ( \mathcal{ C } ) $,
		
		\item 
		for all $ X , Y \in \Ob \mathcal{ C }, \Hom_{ \mathcal{ C} } ( X, Y ) $ a category of 1-morphisms,
		
		\item 
		for all $ X , Y , Z \in \Ob ( \mathcal{C} )$ a composition functor
		\[
			\comp \colon \Hom_{\mathcal{C}} ( Y , Z ) \times \Hom_{ \mathcal{ C } } ( X , Y ) \to \Hom_{ \mathcal{ C } } ( X , Z ),
		\]
		
		\item 
		for all $ X \in \Ob ( \mathcal{C} ) $ an object $ \id_X \in \Ob( \Hom_{ \mathcal{ C } }(  X , X ))$, called identity of $ X $, together with an invertible 2-morphism $v_X = \id_X \circ \id_X \xRightarrow{\sim} \id_X$ in $ \Hom_{ \mathcal{ C } } ( X , X )$ called \textbf{unit constraint},
		
		\item 
		for all $ W , X , Y , Z \in \Ob ( \mathcal{ C } ) $ a natural isomorphism 
		\[
		\begin{tikzcd}[column sep = 2cm]
			\Hom_{ \mathcal{ C } } ( Y , Z ) \times 			\Hom_{ \mathcal{ C } } ( X , Y ) \times 			\Hom_{ \mathcal{ C } } ( W , X ) 
			\ar[r, " {\Hom_{ \mathcal{ C } } ( Y , Z ) \times \comp }"]
			\ar[d, "{ \comp \times \Hom_{ \mathcal{ C } } ( W, X ) }"']
			\ar[d, shift left=4cm, Rightarrow, " \alpha= \alpha_{W , X ,Y , Z} " , "\sim"']
			&
			\Hom_{\mathcal{ C } } ( Y, Z ) \times \Hom_{ \mathcal{ C } } ( W , Y ) 
			\ar[d, " \comp "]
			\\
			\Hom_{\mathcal{ C } } ( X , Z ) \times 
			\Hom_{\mathcal{ C } } ( W , X )
			\ar[r, " \comp "' ]
			&
			\Hom_{\mathcal{ C } } ( W , Z )
		\end{tikzcd}
		\]
		where 
		\[
			\alpha_{ f , g , h } \colon h \circ ( g \circ f )
			\xRightarrow{ \sim }
			( h \circ g ) \circ f \in \Hom_{ \mathcal{ C } } ( W , Z )
		\]
		is an isomorphism called \textbf{associativity constraint},
		
		\item 
		for all $ X , Y \in \Ob ( \mathcal{ C } )$ the following functors are fully faithful
		\begin{align*}
			\Hom_{\mathcal{ C } } ( X , Y )
			&\to 
			\Hom_{\mathcal{ C } } ( X , Y )
			\\
			f
			&\mapsto
			\id_Y \circ f 
			\\
			\Hom_{\mathcal{ C } } ( X , Y ) 
			&\to
			\Hom_{\mathcal{ C } } ( X , Y )
			\\
			f
			&\mapsto 
			f \circ \id_X			
		\end{align*}
	
		\item 
		and for all $ V , W ,X ,Y ,Z \in \Ob ( \mathcal{ C } )$ and all composable 1-morphisms 
		\newline
		$ V \xrightarrow{e} W \xrightarrow{f} X \xrightarrow{ g } Y \xrightarrow{ h } Z $
		the following diagram commutes
		\[
		\begin{tikzcd}[column sep=0.5cm]
			& h \circ ( ( g \circ f ) \circ e )
			\ar[rr, Rightarrow, " \alpha_{ h , g \circ f , e }"]
			&&
			( h \circ ( g \circ f ) ) \circ e 
			\ar[rd, Rightarrow, " \alpha_{ h , g , f } * \id_e"]
			\\
			h \circ ( g \circ ( f \circ e ))
			\ar[ru, Rightarrow, "\id_{h} * \alpha_{g , f , e}"]
			\ar[rrd, Rightarrow," \alpha_{ h , g , f \circ e }"]
			&&&&
			(( h \circ g ) \circ f ) \circ e
			\\
			&&
			( h \circ g ) \circ ( f \circ e )
			\ar[rru, Rightarrow, " \alpha_{ h \circ g , f , e }"]
		\end{tikzcd}
		\]
		which is called the \textbf{Pentagon identity}.
	\end{itemize}
\end{defi}


\begin{exmp}
	Every strict 2-category can be viewed as a 2-category with unit constraints and associativity constraints given by identities.
\end{exmp}

\begin{defi}
	Monoidal categories are 2-categories with a single object.
	That is a 2 category
	$ B \mathcal{ M }$ with $ \Ob ( B \mathcal{ M } ) = \{ * \} $ and $ \mathcal{ M } \coloneqq \twoHom_{B \mathcal{ M} } ( * , * ).$
	The horizontal composition defines the monoidal composition
	\[
		\mathcal{ M } \times \mathcal{ M }
		\xrightarrow{ \otimes }
		\mathcal{M}
	\]
	and there is an associativity constraint $ \alpha_{M_1 , M_2 , M_3 } \colon M_1 \otimes ( M_2 \otimes M_3 ) \isomorphism ( M_1 \otimes M_2 ) \otimes M_3$ in $\mathcal{ M }$.
\end{defi}

\begin{exmp}
	Let $ k $ be a field and $ ( \Vect_k , \otimes_k , k ) $ a monoidal category, the associator is given as:
	\begin{align*}
		\can \colon V_1 \otimes ( V_2 \otimes V_3 ) 
		&\isomorphism 
		( V_1 \otimes V_2 ) \otimes V_3
		\\
		v_1 \otimes ( v_2 \otimes v_3 )
		&\mapsto
		(v_1 \otimes v_2) \otimes v_3
	\end{align*}
\end{exmp}

\begin{exmp}
	Let $ V $ be a category with finite products, then $ ( V , x , * ) $ is a monoidal category, with $ * $ its terminal object and a functor $ - \times - \colon V \times V \to V $, such that $ V_1 \times ( V_2 \times V_3 ) \isomorphism ( V_1 \times V_2 ) \times V_3$.
\end{exmp}

\begin{exmp}	
	The 2-category $\Bim$ of all bimodules has 
	\begin{itemize}
		\item 
		Objects $ \Ob ( \Bim ) $ given by all associative unital rings,
		
		\item 
		for $ R , S \in \Ob ( \Bim ), \twoHom_{\Bim} ( R , S ) \coloneqq \prescript{}{S}{\Mod}_R \simeq L \Fun ( \prescript{}{R}{\Mod}, \prescript{}{S}{\Mod})$,
		
		\item 
		for all $ R , S , T \in \Ob ( \Bim ) $ the horizontal composition is given by the functor 
		\[
		\begin{tikzcd}
			\twoHom_{ \Bim } ( S , T ) \times \twoHom_{ \Bim } ( R , S )
			\ar[r]
			\ar[d, equal]
			&
			\twoHom_{ \Bim } ( R , T )
			\ar[d, equal]
			\\
			\prescript{}{ T }{ \Mod }_S \times \prescript{}{ S }{ \Mod }_R
			\ar[r]
			&
			\prescript{}{ T }{ \Mod }_R 
		\end{tikzcd}
		\] 
		$ ( \prescript{}{ T }{ M }_S , \prescript{}{ S }{ N }_R ) \mapsto (\prescript{}{ T }{ M \otimes_S N }_R ) $,
		
		\item 
		for all $ R \in \Ob ( \Bim ), \id_R = \prescript{}{R}{R}_R \in \prescript{}{R}{\Mod}_R=\twoHom{\Bim}( R , R )$.
		\begin{align*}
			\prescript{}{R}{R} \otimes_R R_R 
			&\isomorphism 
			\prescript{}{R}{R}_R
			\\
			\prescript{}{U}{L} \otimes_T ( M \otimes_S N )
			&\isomorphism_{\can}
			( \prescript{}{U}{L} \otimes_T M_S ) \otimes_S N_R
		\end{align*}
	\end{itemize}
\end{exmp}

\begin{construction}
\label{left_right_unit_constraints}
	Let $\mathcal{ C }$ be a 2-category, there is a fully faithful functor 	
	\begin{align*}
		\twoHom_{ \mathcal{ C } } ( X , Y ) 
		&\xhookrightarrow{f.f.}
		\twoHom_{ \mathcal{ C } } ( X , Y )
		\\ 
		f
		&\mapsto
		\id_Y \circ f		 
	\end{align*}
	for all $ X , Y \in \Ob( \mathcal{ C } )$ and there is a bijection of morphisms called the \textbf{left unit constraint}:
	\begin{align*}
		\id_Y \circ ? \colon \twoHom_{ \mathcal{ C } } ( \id_Y \circ f 	, f )
		&\isomorphism 
		\twoHom_{ \mathcal{ C } } ( X , Y )( \id_Y \circ ( \id_Y \circ f ) , \id_Y \circ f )
	\end{align*}
	\[
	\begin{tikzcd}
		\id_Y \circ f
		\ar[d, Rightarrow, " \exists ! \lambda_f"]
		\\
		f
	\end{tikzcd}
	\mapsto
	\begin{tikzcd}
		\id_Y \circ ( \id_Y \circ f )
		\ar[rr, Rightarrow, " \alpha" , " \sim"']
		\ar[rd, Rightarrow, " \sim"]
		&&
		(\id_Y \circ \id_Y ) \circ f 
		\ar[ld, Rightarrow, "v_Y * \id_f", "\sim"']
		\\
		&
		\id_Y \circ f
	\end{tikzcd}
	\]
	furthermore there is a fully faithful functor:
	\begin{align*}
		\twoHom_{ \mathcal{ C } } ( X , Y ) 
		&\xhookrightarrow{f.f.}
		\twoHom_{ \mathcal{ C }} ( X, Y )
		\\ 
		f
		&\mapsto
		f \circ \id_X		 
	\end{align*}
	as well as a bijection of morphism called the \textbf{right unit constraint}:
	\[
	\begin{tikzcd}
		f \circ \id_X
		\ar[d, Rightarrow, " \exists ! \rho_f"]
		\\
		f
	\end{tikzcd}
	\mapsto
	\begin{tikzcd}
		f \circ ( \id_X \circ \id_X )
		\ar[rr, Rightarrow, " \alpha" , " \sim"']
		\ar[rd, Rightarrow, "\id_f*v_X"' ," \sim"]
		&&
		( f \circ \id_X ) \circ \id_X  
		\ar[ld, Rightarrow, "\sim"']
		\\
		&
		f \circ \id_X 
	\end{tikzcd}
	\]
\end{construction}

\begin{prop}
	Let $ \mathcal{ C } $ be a 2-category. 
	The left and right unit constraints determine natural isomorphisms.
	\[
	\begin{tikzcd}
		\twoHom_{ \mathcal{ C } } ( X , Y )
		\arrow[bend left=50]{rr}[name=U',label=above:$ f \mapsto \id_Y \circ f $]{}
		\arrow[bend right=50]{rr}[name=D',label=below:$ \mathds{ 1 } $]{}
		&&
		\twoHom_{ \mathcal{ C } } ( X , Y )
		\arrow[shorten <=10pt,shorten >=10pt,Rightarrow,to path={(U')-- node[label=right:$ \lambda $] {} (D')}]{}
	\end{tikzcd}
	\]
	\[
	\begin{tikzcd}
		\twoHom_{ \mathcal{ C } } ( X , Y )
		\arrow[bend left=50]{rr}[name=U',label=above:$ f \mapsto f \circ \id_X $]{}
		\arrow[bend right=50]{rr}[name=D',label=below:$ \mathds{ 1 } $]{}
		&&
		\twoHom_{ \mathcal{ C } } ( X , Y )
		\arrow[shorten <=10pt,shorten >=10pt,Rightarrow,to path={(U')-- node[label=right:$ \rho $] {} (D')}]{}
	\end{tikzcd}
	\]
\end{prop}

Lecture 24.4

\begin{proof}
	Exercise.
	Let $ \forall \colon X \to Y, \lambda_f $ is an isomorphism.
	We only prove $ \lambda = ( \lambda_f \colon \id_Y \circ f \Rightarrow f )_{ f \in \Hom_{ \mathcal{C } } ( X , Y ) }$ is a natural transformation.
	Let 
	\[
		f \xRightarrow{\eta} g
	\]
	be a morphism in $ \Hom_{ \mathcal{ C } } ( X, Y )$ 
	\[
	\begin{tikzcd}
		\id_Y \circ f 
		\ar[d, Rightarrow, "\id_Y \circ \eta"']
		\ar[r, Rightarrow, "\lambda_f"]
		&
		f
		\ar[d, Rightarrow, "\eta"]
		\\
		\id_Y \circ g 
		\ar[r, Rightarrow, "\lambda_g"]
		& 
		g
	\end{tikzcd}
	\xrightarrow{ \id_Y \circ - }
	\begin{tikzcd}
		\id_Y \circ ( \id_Y \circ f ) 
		\ar[r,Rightarrow, "\id_Y \circ \lambda_f"]
		\ar[d, Rightarrow, "\id_Y \circ ( \id_Y \circ \eta )"']
		&
		\id_Y \circ f
		\ar[d, Rightarrow, " \id_Y \circ \eta"]
		\\
		\id_Y \circ ( \id_Y \circ g )
		\ar[r, Rightarrow, " \id_Y \circ \lambda_g "]
		&
		\id_Y \circ g
	\end{tikzcd}
	\]
	
	\[
	\begin{tikzcd}
		\id_Y \circ ( \id_Y \circ f )
		\ar[ddd,Rightarrow, " \id_Y \circ ( \id_Y \circ \eta )"']
		\ar[rr, Rightarrow, " \id_Y \circ\lambda_f" ]
		\ar[rd, Rightarrow, "\sim"', "\alpha"]
		&&
		\id \circ f 
		\ar[ddd,Rightarrow, " \id_Y \circ \eta = \id_{ \id_Y } * \eta "]
		\\
		&
		(\id_Y \circ \id_Y ) \circ f 
		\ar[ru,Rightarrow, " v_y * \id_f","\sim"']
		\ar[d, Rightarrow, " ( \id_Y \circ \id_Y) \circ \eta" ]
		\\
		&
		(\id_Y \circ \id_Y ) \circ g
		\ar[rd, Rightarrow, "v_Y * \id_g", "\sim"']
		\\
		\id_Y \circ ( \id_Y \circ g ) 
		\ar[ru, Rightarrow, "\sim", " \alpha"']
		\ar[rr, Rightarrow, " \id_Y \circ \lambda_g "]	
		&&
		\id_Y \circ g
	\end{tikzcd}
	\]
	where the left square commutes by the naturality of the associator constraint, and the top and bottom triangle commute by the left unit constraint.
	For the right square we use the interchange law for composition and the Godement product to obtain $ ( \id_{ \id_Y } * \eta ) \circ ( v_Y * \id_f ) = ( v_\eta * \eta ) = ( v_Y * \id_g ) \circ ( \id_Y \circ \eta ) $.
\end{proof}

\begin{prop}
	Let $ \mathcal{ C } $ be a 2-category and $ X \xrightarrow{ f } Y \xrightarrow{ g } Z $ two composable 1-morphisms in $\mathcal{ C }$.
	Then the following triangle 
	\[
	\begin{tikzcd}
		g \circ ( \id_Y \circ f)
		\ar[rr, Rightarrow, "\alpha", "\sim"']
		\ar[rd, Rightarrow,"\id_g \times \lambda_f"']
		&&
		( g \circ \id_Y ) \circ f 
		\ar[dl, Rightarrow, " \rho_g \times \id_f"]
		\\
		&
		g \circ f 
	\end{tikzcd}
	\]
	commutes.
\end{prop}

\begin{proof}
	Consider the following commutative diagram:
	\[
	\begin{tikzcd}[column sep= 0.1cm, row sep= 1.5cm]
		& 
		g \circ ( ( \id_Y \circ \id_Y ) \circ f )
		\ar[rr, Rightarrow, "\alpha", "\sim"']
		\ar[d, Rightarrow, "v_Y"]
		&&
		( g \circ ( \id_Y \circ \id_Y ) ) \circ f 
		\ar[d, Rightarrow, "u_Y" ]
		\ar[rdd, Rightarrow, "\alpha"]
		\\
		&
		g \circ ( \id_Y \circ f )
		\ar[rr,Rightarrow, "\alpha", "\sim"']
		\ar[d, Rightarrow, "\alpha", "\sim"']
		&&
		( g \circ id_Y ) \circ f
		\\
		g \circ ( \id_Y \circ ( \id_Y \circ f ))
		\ar[ruu, Rightarrow, "\alpha"]
		\ar[ru, Rightarrow, "\lambda_f"']
		\ar[rrd, Rightarrow, "\alpha"]
		& 
		( g \circ \id_Y ) \circ f
		\ar[rru, equal]
		\ar[r, Rightarrow ,"\rho_g", " \sim"']
		&
		g \circ f 
		&
		g \circ ( \id_Y \circ f )
		\ar[ u , Rightarrow, "\alpha" , "\sim"']
		\ar[l, Rightarrow, "\lambda_f"' , "\sim"]
		&
		((g \circ \id_Y ) \circ \id_Y ) \circ f
		\ar[lu, Rightarrow, " \rho_g"]
		\\
		&&
		(g \circ \id_Y ) \circ ( \id_Y \circ f)
		\ar[lu, Rightarrow, " \lambda_f"']
		\ar[ru, Rightarrow, "\rho_g"]
		\ar[rru, Rightarrow, " \alpha"]
		\ar[u, phantom, "{*)}"]
	\end{tikzcd}
	\]
	The triangles commute by applying unit constraints  \ref{left_right_unit_constraints} and the square commute 
	and the squares that include an alpha commute by the associator constraints. 
	The only square that remains is $ *) $, here we use the interchange law for the Godement product to obtain $ ( \rho_g * \id_{f} ) ( \id_{g} * \lambda_f ) = \rho_g * \lambda_f = ( \id_{g} * \lambda_f ) \circ ( \rho_g * \id_f )$.
\end{proof}

\begin{cor}
	Let $ \mathcal{ C } $ be a 2-category and $ X \in \mathcal{ C }$, consider $ \id_X \colon X \to X $.
	Then 
	\begin{align*}
			\lambda_{ \id_X} \colon \id_X \circ \id_X 
			&\xRightarrow{\sim }
			\id_X
			\\
			\rho_{ \id_X} \colon \id_X \circ \id_X
			&\xRightarrow{\sim}
			\id_X
	\end{align*} 
	are both equal to $v_X \colon \id_X \circ \id_X \Rightarrow \id_X$.
\end{cor}

\begin{proof}
	We only do the case $ \lambda_{\id_X}= v_X $.
	By the triangle identity and definition of $ \lambda_{ \id_X} $
	we get that 
	\[
	\begin{tikzcd}	
		\id_X \circ ( \id_X \circ \id_X )
		\ar[rr, Rightarrow, "\alpha", "\sim"']
		\ar[rd, Rightarrow, "\id_X \circ \lambda_{ \id_X}"']
		&&
		(\id_X \circ \id_X ) \circ \id_X
		\ar[dl, Rightarrow," v_X * \id_X"]
		\ar[dl, Rightarrow, bend left=60, " \rho_{\id_X} * \id_X"]
		\\
		&
		\id_X \circ \id_X
	\end{tikzcd}	
	\]
	and thus $ v_X * \id_X = \rho_{\id_X} * \id_X$ which implies that $v_X = \rho_{ \id_X}$ since the composition with the identity is fully faithful.
\end{proof}

\begin{defi}
	Let $\mathcal{ C }$ be a 2-category. The conjugate of $\mathcal{ C } $ is the 2-category $ \mathcal{C}^c = \mathcal{ C }^{co} $ with $\Ob( \mathcal{ C }^c ) = \Ob ( \mathcal{ C } )$ and $ \Hom_{ \mathcal{ C }^c} ( X , Y ) \coloneqq \Hom_{ \mathcal{ C } } ( X , Y )^{ \op}.$
\end{defi}

\begin{defi}
	A $ ( 2 ,1 )$-category is a 2-category such that $ \forall X, Y \in \Ob ( \mathcal{ C } ) , $
	$ \Hom_{ \mathcal{ C } } (X , Y) $ is a groupoid.
\end{defi}

\begin{defi}
	Let $ \mathcal{ C } $ be a 2-category. The coarse homotopy category of $\mathcal{C}$ is the 1-ccategory $h\mathcal{C}$ with $\Ob ( \mathcal{C } ) \coloneqq \Ob(\mathcal{ C } ) $ and with sets of morphisms, $ \Hom_{ h \mathcal{ C } }  ( X , Y ) = \pi_0 ( L \underline{ \Hom }_{ \mathcal{ C } } ( X , Y )) = \pi_0 ( N \underline{\Hom}_{\mathcal{ C } }( X , Y ))$ with the induced composition law, where $L$  is the localisation functor from $ \Cat $ to $\Gpd$.
\end{defi}

\begin{defi}
	Let $\mathcal{ C }$ be a 2-category.
	The pith of $ \mathcal{ C } $ is the 2-category $ \Pith ( \mathcal{ C } )$ with objects $ \Ob ( \Pith ( \mathcal{ C } ) )=\Ob ( \mathcal{ C } )$ with $\underline{ \Hom }_{ \Pith ( \mathcal{ C } ) } ( X, Y ) \coloneqq \underline{\Hom}_{ \mathcal{ C } } ( X , Y )^{ \cong } , $ where $ \underline{\Hom}_{ \mathcal{ C } } ( X , Y )^{ \cong } $ is the maximal subgroupoid of $\underline{\Hom}_{ \mathcal{ C } } ( X , Y ) $.
\end{defi}

\begin{defi}
	The homotopy category of $ \mathcal{ C } $ is $ \hPith ( \mathcal{ C } ) $.
\end{defi}

\subsection{Exercises}

\begin{Exercise}
	In the lecture, we defined the Godement product $ \beta * \alpha \colon G_1 \circ F_1 \Rightarrow G_2 \circ F_1 $ of two natural transformations $ \alpha \colon F_1 \Rightarrow F_2 $ and $ \beta \colon G_1 \Rightarrow G_2 $ for functors $ F_i \colon \mathcal{ A } \to \mathcal{ B } $ and $ G_i \colon \mathcal{ B } \to \mathcal{ C } $ for $ i \in \{ 1 , 2 \} $, pointwise given by $  (\beta * \alpha )_c \coloneqq G_2 ( \alpha_c ) \circ \beta_{ F_1 ( c ) } = \beta_{ F_2 ( c ) } \circ G_1 ( \alpha_c ) $.
	
	\begin{enumerate}[label=(\alph*)]
		\item 
		Show that there is a strict 2-category of categories $ \Cat $ with functor categories as homomorphisms and horizontal composition given by the Godement product, in other words, show that 
		\begin{itemize}
			\item 
			the Godement product is unital, i.e. $\id_{G_1} * \alpha = \alpha$ and $ \beta * \id_{F_1} =\beta,$
			
			\item 
			the Godement product is associative, i.e. $ ( \gamma * \beta) * \alpha = \gamma * ( \beta * \alpha ) $ for $ \gamma \colon H_1 \Rightarrow H_2 \colon \mathcal{ C } \to \mathcal{ D } $,
			
			\item 
			for any two categories $ \mathcal{ C } $ and $ \mathcal{ D } $, we have that both functors 
			\begin{align*}
				\id_{ \mathcal{ C } } \circ ( - ) , ( - ) \circ  \id_{ \mathcal{ D } }\colon \Fun ( \mathcal{ C } , \mathcal{ D } ) 
				&\to 
				\Fun ( \mathcal{ C } , \mathcal{ D } )
				\\
				\eta 
				&\mapsto 
				\id_{\id_{\mathcal{ C }}} * \eta , \eta * \id_{\id_{\mathcal{ C }}}
			\end{align*}
			are the identity functor $ \id_{ \Fun( \mathcal{ C } , \mathcal{ D } ) } $ and
			
			\item 
			vertical and horizontal composition are compatible with each other, i.e. for natural transformations $ \eta \colon F \to F', \eta' \colon F' \to F'', \xi \colon G \to G'$ and $G' \to G''$ of functors $ F , F' , F'' \colon \mathcal{ A } \to \mathcal{ B } $ and $ G , G', G'' \colon \mathcal{ B } \to \mathcal{ C } $ we have that 
			\[
			( \xi' \cdot \xi ) * ( \eta' \cdot \eta ) = ( \xi' * \eta' ) \cdot ( \xi * \eta ) 
			\]
			for $ \cdot $ the pointwise composition of natural transformations.
		\end{itemize}
	\end{enumerate}
	The centre of a category $ \mathcal{ C } $ is defined to be the natural endomorphisms of the identity functor $ \id_{ \mathcal{ C } } \colon \mathcal{ C } \to \mathcal{ C } $. 
	Observe that the centre $ Z ( \mathcal{ C } ) \coloneqq \End_{ \Fun ( \mathcal{ C } , \mathcal{ C } )} ( \id_{ \mathcal{ C } } ) $ is a set that comes both with the pointwise composition $ \cdot $ as well as the Godement product $ * $.
	\begin{enumerate}[resume]
		\item 
		Let $ M $ be a set with two unital binary operations $ \cdot , * \colon M \times M \to M $ such that
		\[
		( a \cdot b ) * ( c \cdot d ) = ( a * c ) \cdot ( b * d ). 
		\]
		The Eckmann-Hilton argument states that then both products agree and $ ( M , \cdot , e ) $ is an abelian monoid. Prove that this assertion is true.
		
		\item 
		Conclude that the centre of a category is an abelian monoid with either product. 
		
	\end{enumerate}
\end{Exercise} 

\begin{Exercise}
	Fix three groupoids $ \mathcal{ A } , \mathcal{ B } $ and $ \mathcal{ C } $ together with functors $ F \colon \mathcal{ A } \to \mathcal{ C } $ and $ G \colon \mathcal{ B } \to \mathcal{ C } $.
	Recall that we defined the 2-pullback $ \mathcal{ A } \times_{ \mathcal{ C } }^{ ( 2 ) } \mathcal{ B } $ with objects $ ( A \in \mathcal{ A } , B \in \mathcal{ B } , \varphi \colon FA \to GB )$.
	
	\begin{enumerate}[label=(\alph*)]
		\item 
		Confirm that $ \mathcal{ A } \times_{ \mathcal{ C } }^{ ( 2 ) } \mathcal{ B } $ is a groupoid.
		
		\item 
		Show that $ \Fun ( \mathcal{ D } , \mathcal{ C } ) $ is a groupoid for any groupoid $ \mathcal{ D } $.
		
		\item 
		Show that the canonical morphism 
		\[
		\Fun ( - , \mathcal{ A } \times_{ \mathcal{ C } }^{ ( 2 ) } \mathcal{ B } ) 
		\to
		\Fun ( - , \mathcal{ A } ) \times_{ \Fun ( - , \mathcal{ C } )} ^{(2)} \Fun ( - , \mathcal{ B } )
		\]
		is an isomorphism of functors $ \Gpd^{ \op } \to \Gpd$.
	\end{enumerate}
\end{Exercise}

\begin{Exercise}
	Fix a monoid $ M = ( M , \otimes , e ) $ and view it as a category $ \mathcal{ M } $ whose objects are the elements of $ M $ with only identity morphisms.
	Define the monoidal category of $ M $ to be the 2-category $ \underline{ BM } $ with one object $ \star $ and $ \End_{ \underline{ BM } } ( \star ) \coloneqq \mathcal{ M } $.
	Show that $ \underline{ BM } $ is indeed a 2-category with horizontal composition functor
	\[
	\otimes \colon \mathcal{ M } \times  \mathcal{ M } = M \times M \to M = \mathcal{ M }
	\]
	which is defined on morphisms in the unique way $ \id_m \otimes \id_n = \id_{ m \times n } $.
	How does this construction compare to an ordinary 1-category with one object viewed as a 2-category with only identity 2-morphisms?
\end{Exercise}

\begin{Exercise}
	A strict monoidal category is a category $ \mathcal{ C } $ together with a functor $ - \times - \colon \mathcal{ C } \times \mathcal{ C } \to \mathcal{ C } $ and distinguished element $ \mathds{ 1 } \in \mathcal{ C } $ such that its delooping $ B \mathcal{ C } $ with single object $ \star $ and $ \End_{ B \mathcal{ C } } $ defines a strict 2-category with horizontal composition $ \otimes $ and $ \id_* \coloneqq \mathds{ 1 } $.
	\begin{enumerate}[label=(\alph*)]
		\item 
		Spell out the explicit axioms of a strict monoidal category without reference to a 2-category.
	\end{enumerate}
	
	Consider the category $ \mathcal{ V } \coloneqq \mathcal{ V } ( \mathbb{ K } ) $ which has as objects the natural numbers $ \mathbb{ N }_0 $ and homomorphisms from $ n $ to $ m $ given by $ n \times m $-matrices over a fixed field $ \mathbb{ K } $.
	Equip it with a monoidal structure via
	\begin{align*}
		- \otimes - \colon \mathcal{ V } \otimes \mathcal{ V } 
		&\to 
		\mathcal{ V } 
		\\
		( n , m ) 
		&\mapsto
		mn
		\\
		( A , B ) 
		&\mapsto
		A \otimes B
	\end{align*}
	where for matrices $ A $ and $ B $ we denote by $ A \otimes B $ their Kronecker product, i.e. the matrix $ ( a_{ij}B)_{ij}$ for $ A = ( a_{ij})_{ij}$.
	
	\begin{enumerate}[resume, label=(\alph*)]
		\item 
		Confirm that $ \mathcal{ V } $ is a strict monoidal category with unit element $ \mathds{ 1 } \coloneqq 1 $.
		
		\item 
		What is the difference between $ \mathcal{ V } $ and the category $ \vect_{fd} ( \mathbb{ K } ) $ of finite dimensional $ \mathbb{ K } $-vector spaces equipped with the ordinary tensor product?
		Is $ \vect_{fd} ( \mathbb{ K } ) $ with the ordinary tensor product a strict monoidal category.
	\end{enumerate}
\end{Exercise}

\begin{Exercise}
	Fix a strict 2-category $ \mathcal{ C } = ( \mathcal{ C } , \Hom_{ \mathcal{ C } } ( \circ , * ) ) $.
	\begin{enumerate}[label=(\alph*)]
		\item 
		Show that any object $ c \in \mathcal{ C } $ induces a strict 2-functor
		\[
		\Hom_{ \mathcal{ C } } ( c , - ) \colon \mathcal{ C } \to \underline{\Cat}
		\]
		from $\mathcal{ C } $ to the 2-category of categories $ \underline{ \Cat } $.
		
		\item 
		Define the composition law of $ \mathcal{ C }^{ \op } $ with $ \Hom_{ \mathcal{ C }^{\op }} ( c , d ) \coloneqq \Hom_{\mathcal{ C } } ( d, c ) $ and confirm that your definition yields a strict 2-category.
		
		\item 
		Show that $ \Hom_{ \mathcal{ C } } ( - , c ) \colon \mathcal{ C }^{ \op } \to \underline{ \Cat }$ is also a 2-functor by arguing that 
		\[
		\Hom_{\mathcal{ C } } ( - , c ) = \Hom_{ \mathcal{ C }^{\op} } ( c , - )
		\]
		for all $ c \in \mathcal{ C } $.
	\end{enumerate}
\end{Exercise}

\begin{Exercise}
	Fix a group $ G = ( G , \cdot , 1 ) $ and an abelian group $ \Gamma = ( \Gamma , + , 0 ) $ together with a group homomorphism $ \chi \colon G \to \Aut_{\Grp} ( \Gamma ) $.
	Instead of $ \chi ( g ) ( \gamma ) $ we will also write $ g ( \gamma ) $.
	\begin{enumerate}[label=(\alph*)]
		\item 
		Show that there is a cochain complex of abelian groups $ C^{ \bullet } ( G , \Gamma ) $ with $ C^n ( G , \Gamma ) \coloneqq \Hom_{ \Set } ( G^n , \Gamma ) $ with $ G^0 \coloneqq \{ 1\}$ and differential 
		\begin{align*}
			d^n ( \varphi ) ( g_1 , \dotsc ,  g_{n+1} ) 
			\coloneqq&
			g_1 ( \varphi ( g_2 , \dotsc ,  g_{n+1} ) ) + ( - 1 )^{ n + 1 } \varphi ( g_1 , \dotsc ,  g_n ) 
			\\
			&+ \sum_{i = 1}^n ( - 1 )^i \varphi ( g_1 , \dotsc ,  g_{ i-1 } , g_ig_{i+1} , g_{i+2}, \dotsc ,  g_{n+1} )
		\end{align*}
		for $ n \in \mathbb{ N }_0 $ and $ 0 $ otherwise. 
		In other words, show that $ d^{n+1} \circ d^n = 0 $ for all $ n \in \mathbb{ Z } $.
	\end{enumerate}
	
	We call a morphism $ \varphi \colon G^n \to \Gamma  $ an n-cocycle if $ d^n ( \varphi ) = 0 $.
	The $ n $-th cohomology $ H^n ( G , \Gamma ) $ of $ G $ with coefficients in $ \Gamma $ is defined as the quotient of $ \ker ( d^n ) $ by the image of $d^{n-1}$.
	
	\begin{enumerate}[resume, label=(\alph*)]
		\item 
		Describe the 0-th cohomology $ H^0 ( G , \Gamma ) $.
	\end{enumerate}
	
	Consider now a category $ \mathcal{ C } = \mathcal{ C } ( G , \Gamma ) $ whose objects are the elements of $ G $, any object has endomorphisms $ \Gamma $ and no other morphisms exist.
	
	\begin{enumerate}[resume, label=(\alph*)]
		\item 
		Show that there is a functor $ \otimes \colon \mathcal{ C } \times \mathcal{ C } \to \mathcal{ C } $ which on objects is just the multiplication of $ G $ and for morphisms $ \gamma \otimes \delta \coloneqq \gamma + x ( \delta ) $ for $ ( \gamma, \delta ) \colon ( x , y ) \to ( x, y ) $.
		
		\item 
		Argue that upgrading $ \mathcal{ C } $ to a monoidal category with tensor product (horizontal multiplication) $ \otimes $ and unit constraint $ 0 \colon 1 \times 1 \to 1 $ is equivalent to choosing a $ 3 $-cocycle $ \alpha \colon G \times G \times G \to \Gamma$.
	\end{enumerate}
\end{Exercise}

\begin{Exercise}
	Let $ \mathcal{ C } $ be a small category with pullbacks.
	
	\begin{enumerate}[label=(\alph*)]
		\item 
		Show that for objects $ x , y \in \mathcal{ C } $ there is a category $ \Span ( \mathcal{ C } ) ( x , y ) $ with objects
		\[
		\coprod_{ z \in \mathcal{ C } } ( \Hom_{ \mathcal{ C } }  ( z , x ) \times \Hom_{ \mathcal{ C } } ( z , y ) )
		\]
		and morphism sets
		\[
		\Hom_{ \Span ( \mathcal{ C } ) ( x , y ) } ( x \xleftarrow{f} z \xrightarrow{g} y,
		x \xleftarrow{f'} z'
		\xrightarrow{g'} y )
		\coloneqq
		\{ \eta \in \Hom_{ \mathcal{ C }} ( z , z' ) \mid f = f' \circ \eta \wedge g = g' \circ \eta \}
		\]
		with the induced comosition law from $ \mathcal{ C } $.
		
		\item 
		Show that the pullbacks in $ \mathcal{ C } $ induce functors 
		\[
		\mu_{ x , y , z } \colon \Span ( \mathcal{ C } ) ( y , z ) \times \Span ( \mathcal{ C } ) ( x , y )  \to 
		\Span ( \mathcal{ C } ) ( x , z ) 
		\]
		for every $ x , y , z \in \mathcal{ C } $.
		
		\item 
		Explain how the above data assembles into a 2-category $ \underline{ \Span } ( \mathcal{ C } )$. Justify that it is not a strict 2-category.
	\end{enumerate}
\end{Exercise}

\begin{Exercise}
	Consider the final category $ [ 0 ] $ which has one object whose only endomorphism is its identity.
	We may canonically view it as a (strict) 2-category $ \underline{ [ 0 ] } $ with only an identity 2-morphism.
	Let $ \mathcal{ M } \coloneqq( \mathcal{ M } , \otimes , \mathds{ 1 } ) $ be a monoidal category with delooping 2-category $ B \mathcal{ M } $. 
	An algebra object of $ \mathcal{ M } $ is defined as a lax functor $ A \colon \underline{ [ 0 ] } \to B \mathcal{ M } $. 
	\begin{enumerate}[label=(\alph*)]
		\item 
		Give the definition of an algebra object of $ \mathcal{ M } $ without reference to 2-categories.
		
		\item 
		Justify the name algebra object by computing algebra objects of the monoidal category $ \vect_{ \mathbb{ K } } \coloneqq ( \vect_{ \mathbb{ K } }, \otimes , \mathbb{ K } ) $ for a field $ \mathbb{ K } $.
		
		\item 
		Describe algebra objects in the following monoidal categories.
		\[
		\begin{tikzcd}
			1. ( \set , \times , ¸\{ \star \} ) 
			&
			2. ( \Ab , \otimes, \mathbb{ Z } )
			&
			3. ( \Cat , \times, [ 0 ] )
		\end{tikzcd}
		\]
		
		\item 
		Sketch that the algebra objects in $ \mathcal{ M } $ form themselves a monoidal category $ \Alg ( \mathcal{ M } ) $.
		
		\item 
		Argue that the algebra objects of $ \Alg ( \mathcal{ M } ) $ are 'commutative algebra objects'.
		
		\item 
		Give the definition of a coalgebra object.
		
		\item 
		Show that for a group $ G $, the group algebra $ \mathbb{ K } G $ carries the structure of a coalgebra with comultiplication $ g \mapsto g \otimes g $ and counit $ g \mapsto 1_{ \mathbb{K} }$ in the monoidal category $ \vect_{ \mathbb{ K } }$.
		
		\item 
		What kind of additional structure on $ \mod ( \mathbb{ K } G ) $ could the coalgebra structure above yield?
	\end{enumerate}
\end{Exercise}

