Lecture 17.04

\begin{defi}
	A strict 2-category $ \mathcal{ C } $ 
	consists of:
	\begin{enumerate}
		\item 
		A class $ \Ob ( \mathcal{C} ) $ of objects of $ \mathcal{C} $
		
		\item 
		For all $ X , Y \in \Ob( \mathcal{ C } ) $ a category $ \Hom_{ \mathcal{ C } } ( X , Y )$ whose objects 
		$f \colon X \to Y $ are called $1$-morphisms and whose morphisms $ \alpha \colon f \Rightarrow g $ are called 2-morphisms, with a vertical composition of 2-morphisms, that 
		is associative, and unital.
		
		\item 
		For all $ X , Y , Z \in \Ob( \mathcal{C} ) $ a horizontal composition functor
		\[
		-\circ- \colon \Hom_{ \mathcal{ C } } ( Y , Z ) \times \Hom_{ \mathcal{ C } } ( X , Y ) \to \Hom_{ \mathcal{ C } } ( X , Z )
		\]
		that is compatible with the vertical composition, in the following way 
		\[
		\begin{tikzcd}
			X
			\arrow[bend left=50]{rr}[name=U',label=above:$f$]{}
			\arrow[bend right=50]{rr}[name=D',label=below:$f'$]{}
			&&
			Y
			\arrow[shorten <=10pt,shorten >=10pt,Rightarrow,to path={(U')-- node[label=right:$ \alpha $] {} (D')}]{} 
			\arrow[bend left=50]{rr}[name=U'',label=above:$g$]{}
			\arrow[bend right=50]{rr}[name=D'',label=below:$g'$]{}
			&&
			Z
			\arrow[shorten <=10pt,shorten >=10pt,Rightarrow,to path={(U'')-- node[label=right:$\beta$] {} (D'')}]{} 
		\end{tikzcd}
		\mapsto
		\begin{tikzcd}
			X
			\arrow[bend left=50]{rr}[name=U',label=above:$g \circ f$]{}
			\arrow[bend right=50]{rr}[name=D',label=below:$g' \circ f'$]{}
			&&
			Z
			\arrow[shorten <=10pt,shorten >=10pt,Rightarrow,to path={(U')-- node[label=right:$ \beta * \alpha $] {} (D')}]{} 
		\end{tikzcd} 	
		\]
		
		\item 
		Functoriality of horizontal composition:
		\[
		\begin{tikzcd}
			X
			\arrow[bend left=50]{rr}[name=U',label=above:$f$]{}
			\arrow[bend right=50]{rr}[name=D',label=below:$f$]{}
			&&
			Y
			\arrow[shorten <=10pt,shorten >=10pt,Rightarrow,to path={(U')-- node[label=right:$ \id_f $] {} (D')}]{} 
			\arrow[bend left=50]{rr}[name=U'',label=above:$g$]{}
			\arrow[bend right=50]{rr}[name=D'',label=below:$g$]{}
			&&
			Z
			\arrow[shorten <=10pt,shorten >=10pt,Rightarrow,to path={(U'')-- node[label=right:$ \id_g $] {} (D'')}]{} 
		\end{tikzcd}
		\mapsto
			\begin{tikzcd}
			X
			\arrow[bend left=50]{rr}[name=U',label=above:$g \circ f$]{}
			\arrow[bend right=50]{rr}[name=D',label=below:$g \circ f$]{}
			&&
			Z
			\arrow[shorten <=10pt,shorten >=10pt,Rightarrow,to path={(U')-- node[label=right:$ \id_{ g \circ f } $] {} (D')}]{} 
		\end{tikzcd} 
		\]
		
		\item 
		There is a composition
		\[
		\begin{tikzcd}[column sep=huge]	
			X
			\arrow[bend left=90]{r}[name=U,label=above:$f_1$]{}
			\arrow[bend right=90]{r}[name=D,label=below:$f_3$]{}
			\arrow{r}[name=M]{}
			\arrow{r}[label=below:$f_2$, pos=0.2]{}
			&
			Y
			\arrow[shorten <=5pt,shorten >=5pt,Rightarrow,to path={(U) -- node[label=right:$\alpha$] {} (M)}]{}
			\arrow[shorten <=5pt,shorten >=5pt,Rightarrow,to path={(M) -- node[label=right:$\gamma$] {} (D)}]{}
			\arrow[bend left=90]{r}[name=U',label=above:$g_1$]{}
			\arrow[bend right=90]{r}[name=D',label=below:$g_3$]{}
			\arrow{r}[name=M']{}
			\arrow{r}[label=below:$g_2$, pos=0.2]{}
			&
			Z
			\arrow[shorten <=5pt,shorten >=5pt,Rightarrow,to path={(U') -- node[label=right:$\beta$] {} (M')}]{}
			\arrow[shorten <=5pt,shorten >=5pt,Rightarrow,to path={(M') -- node[label=right:$\delta$] {} (D')}]{}    	
		\end{tikzcd}	
		\mapsto 
		\begin{tikzcd}
			X
			\arrow[bend left=50]{rr}[name=U',label=above:$g_1 \circ f_1$]{}
			\arrow[bend right=50]{rr}[name=D',label=below:$g_3 \circ f_3$]{}
			&&
			Z
			\arrow[shorten <=10pt,shorten >=10pt,Rightarrow,to path={(U')-- node[label=right:$ \eta $] {} (D')}]{}
		\end{tikzcd}	
		\]
		where $\eta $ is the Godement product $( \delta * \gamma ) \circ ( \beta * \alpha ) $.
	\end{enumerate}

The above data should satisfy the following axioms:
\begin{itemize}
	\item 
	(Unitality) 
	For all $ X \in \Ob ( \mathcal{ C } ) \exists \id_X \in \Ob ( \Hom_{ \mathcal{C } } ( X, X ))$ an identity 1-morphism such that $ \forall Y \in \Ob( \mathcal{ C } ) $, the functor 
	\begin{align*}
		\Hom_{ \mathcal{ C } } ( X , Y ) \times \mathds{ 1 } &\to \Hom_{ \mathcal{ C } } ( X , Y ) \times \Hom_{ \mathcal{ C } } ( X , X ) \to \Hom_{ \mathcal{C } } ( X, Y )
		\\
		f &\mapsto (f, \id_X ) \mapsto f \circ \id_X
	\end{align*}
is equal to the identity functor $ \Hom_{ \mathcal{ C } } ( X , Y ) \to \Hom_{ \mathcal{ C }  } ( X, Y )$.
Similarly the functor
	\begin{align*}
		\mathds{ 1 } \times \Hom_{ \mathcal{ C } } ( Y , X ) &\to \Hom_{ \mathcal{ C } } ( X , X ) \times \Hom_{ \mathcal{ C } } ( Y , X ) \to \Hom_{ \mathcal{C } } ( Y, X )
		\\
		g &\mapsto \id_X \circ g = g
	\end{align*}
is equal to the identity functor$ \Hom_{ \mathcal{ C } } ( Y , X ) \to \Hom_{ \mathcal{ C }  }(  Y , X )$.

	\item 
	(Associativity) 
	For all $ W , X , Y , Z \in \Ob ( \mathcal{ C } ) $: the following square of functors commutes \underline{strictly}.
	\[
	\begin{tikzcd}[column sep= 3cm]
		\Hom_{ \mathcal{ C } } ( Y , Z ) \times \Hom_{ \mathcal{ C } } ( X , Y ) \times \Hom_{ \mathcal{ C } } ( W , X )
		\ar[r, " {\Hom_{ \mathcal{ C  } } ( Y , Z ) \times ( - \circ - ) }"]
		\ar[d, "{ ( - \circ - ) \times \Hom_{ \mathcal{ C } } ( W , X ) }"]
		&
		\Hom_{\mathcal{ C } } ( Y , Z ) \times \Hom_{ \mathcal{ C } } ( W , X )
		\ar[d, " {( - \circ - ) }"]
		\\
		\Hom_{ \mathcal{ C }}]( X , Z ) \times \Hom_{\mathcal{ C } } ( W , X )
		\ar[r, "{ ( - \circ -) }"]
		&
		\Hom_{ \mathcal{ C } }  ( W , Z )
	\end{tikzcd}
	\]
\end{itemize}
\end{defi}

\begin{defi}
	A strict monoidal category is a strict 2-category with a single object, that is the following data.
	\begin{enumerate}
		\item 
		A strict 2-category $B \mathcal{ M }$ with $\Ob ( B \mathcal{ M } ) = \{ * \} $.
		
		\item 
		A category $ \mathcal{ M } \coloneqq \Hom_{ B \mathcal{ M }  } ( * , * ).$
		
		\item 
		A monoidal composition $ - \otimes - \colon \mathcal{ M } \times \mathcal{ M } \to \mathcal{ M } $, that fullfills 
		the following axioms.
		
		\begin{itemize}
			\item 
			(Unitality)
			There exists $ \mathds{ 1 }_{ \mathcal{ M } } \in \Ob( \mathcal{ M } ) = \Ob ( Hom_{ B \mathcal{M} } ( * , * ) )$ such that the functor 
			\begin{align*}
				\mathcal{ M } 
				& \to
				\mathcal{ M }
				\\
				M
				&\mapsto 
				M \otimes \mathds{1}_{\mathcal{ M } }
			\end{align*}
			is the identity, that is
			gives equalities $ M \otimes  \mathds{1}_{ \mathcal{M} } = M = \mathds{1}_{\mathcal{ M } } \otimes M$.
			
			\item 
			(Associativity)
			The following square commutes
			\[
			\begin{tikzcd}[column sep = 2cm]
				\mathcal{M} \times \mathcal{ M } \times  \mathcal{ M }
				\ar[r, " \id_{ \mathcal{ M } } \times  ( - \circ - )"]
				\ar[d, " ( - \otimes - ) \times \id_{ \mathcal{ M }}"']
				&
				\mathcal{ M  } \times \mathcal{ M }
				\ar[d, "\otimes "]
				\\
				\mathcal{ M } \times \mathcal{ M }
				\ar[r, " - \otimes - " ]
				&
				\mathcal{ M } 
			\end{tikzcd}
			\]
			that is for all $ M_1 , M_2 , M_3 \in \mathcal{ M } $ 
			it holds that $ M_1 \otimes  ( M_2 \otimes M_3 ) = ( M_1 \otimes M_2 ) \otimes M_3$.
		\end{itemize}
	\end{enumerate}		
\end{defi}

\begin{defi}
	Let $ \mathcal{ C } , \mathcal{ D } $  be strict 2-categories.
	A strict 2-functor $ F \colon \mathcal{ C } \to \mathcal{ D } $ consists of 
	\begin{enumerate}
		\item 
		A map $ F \colon \Ob ( \mathcal{ C } ) \to \Ob ( \mathcal{ D } )$  sending an object $ X $ to $ F ( X ) $.
		
		\item 
		For all $ X , Y \in \Ob ( \mathcal{ C } ) $ a functor $ F_{XY} \colon \Hom_{ \mathcal{ C } } ( X , Y ) \to \hom_{ \mathcal{ D }} ( F X , F Y ) $ such that the following hold 
		\begin{itemize}
			\item 
			(Unitality)
			$ \forall X \in \Ob ( \mathcal{ C } ) F ( \id_X ) = \id_{ FX } \in \Hom_{ \mathcal{ D } } ( FX, FX )$
				
			\item 
			(Composition)
			Let $ X, Y , Z \in \Ob ( \mathcal{C}) $
			\[
			\begin{tikzcd}
				\Hom_{ \mathcal{ C } } ( Y, Z ) \times \Hom_{ \mathcal{ C } }( X , Y )
			\end{tikzcd}
			\]
		\end{itemize}
	\end{enumerate}
	Let $ \mathcal{ C } \colon $ 2-category then the opposite category $ \mathcal{ C }^{ \op } $ is also a strict 2-category.
\end{defi}

Lecture 22.04

\begin{rmk}
	Every ordinary 1-category $ \mathcal{ C } $  can be viewed as a strict 2-category with 
	\begin{itemize}
		\item 
		Forall $ X , Y \in \Ob ( \mathcal{ C } ) , \Hom_{\mathcal{ C } } ( X , Y ) \coloneqq \Hom_{ \mathcal{C } }  ( X , Y ).$
		
		\item 
		Horizontal composition = composition in $ \mathcal{ C } $
		
		\item 
		For all $ X \in \Ob ( \mathcal{ C } )\label{key} $ it holds that $ \id_X \in \Hom_ { \mathcal{ C } } ( X , X ) = \Hom_{ \mathcal{ C } } ( X , X ) $ is the identity morphism in the original catgory $ \mathcal{ C }.$
			
		\item 
		Conversely every strict 2-catgory $ \mathcal{ C } $ has an underlying ordinary category $ \mathcal{ C }_0 $ with $ \Ob ( \mathcal{ C }_0 ) \coloneqq \Ob ( \mathcal{ C } )$
		and $ \forall X , Y \in \Ob ( \mathcal{ C } ) = \Ob ( \mathcal{ C } ), \Hom_{ \mathcal{ C }_0 } ( X , Y  ) \coloneqq \Ob ( \Hom_{ \mathcal{ C } }( X ,Y )).$
		The composition law in $ \mathcal{ C }_0 $ is horizontal composition of 1-morphisms in $ \mathcal{ C } $, this composition is associative since $ \mathcal{ C } $ is a strict 2-category.  
	\end{itemize}
\end{rmk}

\begin{defi}
	A 2-category (bicategory ) $ \mathcal{ C }$ consists of a 2-category $ \mathcal{ C } $
	\begin{itemize}
		\item 
		A class of objects of $ \mathcal{ C } $,
		
		\item 
		$\forall X , Y \in \Ob \mathcal{ C }, \Hom_{ \mathcal{ C} } ( X, Y ) $ a category of 1-morphisms,
		
		\item 
		for all $ X , Y , Z \in \Ob ( \mathcal{C} )$ a composition functor
		\[
			\comp \colon \Hom_{\mathcal{C}} ( Y , Z ) \times \Hom_{ \mathcal{ C } } ( X , Y ) \to \Hom_{ \mathcal{ C } } ( X , Z ),
		\]
		
		\item 
		for all $ X \in \Ob ( \mathcal{C} ) $ an object $ \id_X \in \Ob( \Hom_{ \mathcal{ C } }(  X , X ))$ together with an invertible 2-morphism $v_X = \id_X \circ \id_X \xRightarrow{\sim} \id_X$ in $ \Hom_{ \mathcal{ C } } ( X , X )$ called unit constraints,
		
		\item 
		for all $ W , X , Y , Z \in \Ob ( \mathcal{ C } ) $ a natural isomorphism 
		\[
		\begin{tikzcd}[column sep = 2cm]
			\Hom_{ \mathcal{ C } } ( Y , Z ) \times 			\Hom_{ \mathcal{ C } } ( X , Y ) \times 			\Hom_{ \mathcal{ C } } ( W , X ) 
			\ar[r, " {\Hom_{ \mathcal{ C } } ( Y , Z ) \times \comp }"]
			\ar[d, "{ \comp \times \Hom_{ \mathcal{ C } } ( W, X ) }"']
			\ar[d, shift left=4cm, Rightarrow, " \alpha= \alpha_{W , X ,Y , Z} " , "\sim"']
			&
			\Hom_{\mathcal{ C } } ( Y, Z ) \times \Hom_{ \mathcal{ C } } ( W , Y ) 
			\ar[d, " \comp "]
			\\
			\Hom_{\mathcal{ C } } ( X , Z ) \times 
			\Hom_{\mathcal{ C } } ( W , X )
			\ar[r, " \comp "' ]
			&
			\Hom_{\mathcal{ C } } ( W , Z )
		\end{tikzcd}
		\]
		where 
		\[
			\alpha_{ f , g , h } \colon h \circ ( g \circ f )
			\xRightarrow{ \sim }
			( h \circ g ) \circ f \in \Hom_{ \mathcal{ C } } ( W , Z )
		\]
		
		\item 
		For all $ X , Y \in \Ob ( \mathcal{ C } )$ the following functors are fully faithful:
		\begin{align*}
			\Hom_{\mathcal{ C } } ( X , Y )
			&\to 
			\Hom_{\mathcal{ C } } ( X , Y )
			\\
			f
			&\mapsto
			\id_Y \circ f 
			\\
			\Hom_{\mathcal{ C } } ( X , Y ) 
			&\to
			\Hom_{\mathcal{ C } } ( X , Y )
			\\
			f
			&\mapsto 
			f \circ \id_X			
		\end{align*}
	
		\item 
		For all $ V , W ,X ,Y ,Z \in \Ob ( \mathcal{ C } )$ and all composable 1-morphisms $ V \xrightarrow{e} W \xrightarrow{f} X \xrightarrow{ g } Y \xrightarrow{ h } Z $
		we have the Pentagon-identity 
		
		\begin{tikzcd}[column sep=0.5cm]
			& h \circ ( ( g \circ f ) \circ e )
			\ar[rr, Rightarrow, " \alpha_{ h , g \circ f , e }"]
			&&
			( h \circ ( g \circ f ) ) \circ e 
			\ar[rd, Rightarrow, " \alpha_{ h , g , f } * \id_e"]
			\\
			h \circ ( g \circ ( f \circ e ))
			\ar[ru, Rightarrow, "\id_{h} * \alpha_{g , f , e}"]
			\ar[rrd, Rightarrow," \alpha_{ h , g , f \circ e }"]
			&&&&
			(( h \circ g ) \circ f ) \circ e
			\\
			&&
			( h \circ g ) \circ ( f \circ e )
			\ar[rru, Rightarrow, " \alpha_{ h \circ g , f , e }"]
		\end{tikzcd}
	\end{itemize}
\end{defi}


\begin{exmp}
	Every strict 2-category can be viewed as a 2-category with the identity unit and associativity constraints.
\end{exmp}

\begin{defi}
	Monoidal categories are 2-categories with a single object.
	That is a 2 category
	$ B \mathcal{ M }$ with $ \Ob ( B \mathcal{ M } ) = \{ * \} $ and $ \mathcal{ M } \coloneqq \twoHom_{B \mathcal{ M} } ( * , * ).$
	The horizontal composition defines the monoidal composition
	\[
		\mathcal{ M } \times \mathcal{ M }
		\xrightarrow{ \otimes }
		\mathcal{M}
	\]
	that is $ \alpha_{M_1 , M_2 , M_3 } \colon M_1 \otimes ( M_2 \otimes M_3 ) \isomorphism ( M_1 \otimes M_2 ) \otimes M_3$ in $\mathcal{ M }$.
\end{defi}

\begin{exmp}
	Let $ k $ be a field and $ ( \Vect_k , \otimes_k , k )$ a monoidal category, that is we have 
	\begin{align*}
		can \colon V_1 \otimes ( V_2 \otimes V_3 ) 
		&\isomorphism 
		( V_1 \otimes V_2 ) \otimes V_3
		\\
		v_1 \otimes ( v_2 \otimes v_3 )
		&\mapsto
		(v_1 \otimes v_2) \otimes v_3
	\end{align*}
\end{exmp}

\begin{exmp}
	Let $ V $ be a category with finite products, then $ ( V , x , * ) $ is a monoidal category, with $ * $ its terminal object.
	We need a functor $ - \times - \colon V \times V \to V $, such that $ V_1 \times ( V_2 \times V_3 ) \isomorphism ( V_1 \times V_2 ) \times V_3$.
\end{exmp}

\begin{exmp}	
	The 2-category $\Bim$ of all bimodules has 
	\begin{itemize}
		\item 
		Objects $ \Ob ( \Bim ) $ given by all associative unital rings,
		
		\item 
		for $ R , S \in \Ob ( \Bim ), \twoHom_{\Bim} ( R , S ) \coloneqq \prescript{}{S}{\Mod}_R \simeq L \Fun ( \prescript{}{R}{\Mod}, \prescript{}{S}{\Mod})$,
		
		\item 
		for all $ R , S , T \in \Ob ( \Bim ) $ the horizontal composition is given by the functor 
		\[
		\begin{tikzcd}
			\twoHom{\Bim}(S,T) \times \twoHom_{\Bim}(R,S)
			\ar[r]
			\ar[d, equal]
			&
			\twoHom{\Bim}( R , T )
			\ar[d, equal]
			\\
			\prescript{}{T}{\Mod}_S \times \prescript{}{S}{\Mod}_R
			\ar[r]
			&
			\prescript{}{T}{\Mod}_R 
		\end{tikzcd}
		\] 
		$( \prescript{}{T}{M}_S , \prescript{}{S}{N}_R ) \mapsto (\prescript{}{T}{M \otimes_S N}_R)$
		
		\item 
		For all $ R \in \Ob ( \Bim ), \id_R = \prescript{}{R}{R}_R \in \prescript{}{R}{\Mod}_R=\twoHom{\Bim}( R , R )$.
		\begin{align*}
			\prescript{}{R}{R} \otimes_R R_R 
			&\isomorphism 
			\prescript{}{R}{R}_R
			\\
			\prescript{}{U}{L} \otimes_T ( M \otimes_S N )
			&\isomorphism_{\can}
			( \prescript{}{U}{L} \otimes_T M_S ) \otimes_S N_R
		\end{align*}
	\end{itemize}
\end{exmp}

\begin{construction}
\label{left_right_unit_constraints}
	Let $\mathcal{ C }$ be a 2-category $ \forall X , Y \in \Ob ( \mathcal{ C } ) $, there is a fully faithful functor 	
	\begin{align*}
		\twoHom{\mathcal{ C } } ( X , Y ) 
		&\xhookrightarrow{f.f.}
		\twoHom{ \mathcal{ C }} ( X, Y )
		\\ 
		f
		&\mapsto
		\id_Y \circ f		 
	\end{align*}
	for all $ X , Y \in \Ob( \mathcal{ C } )$ and there is a bijection of morphisms called the left unit constraint
	\begin{align*}
		\id_Y \circ ? \colon \twoHom{\mathcal{ C} } ( \id_Y \circ f 	, f )
		&\isomorphism 
		\twoHom{ \mathcal{C } } ( X , Y )( \id_Y \circ ( \id_Y \circ f ) , \id_Y \circ f )
	\end{align*}
	\[
	\begin{tikzcd}
		\id_Y \circ f
		\ar[d, Rightarrow, " \exists ! \lambda_f"]
		\\
		f
	\end{tikzcd}
	\mapsto
	\begin{tikzcd}
		\id_Y \circ ( \id_Y \circ f )
		\ar[rr, Rightarrow, " \alpha" , " \sim"']
		\ar[rd, Rightarrow, " \sim"]
		&&
		(\id_Y \circ \id_Y ) \circ f 
		\ar[ld, Rightarrow, "v_Y * \id_f", "\sim"']
		\\
		&
		\id_Y \circ f
	\end{tikzcd}
	\]
	Furthermore there is a fully faithful functor.
	\begin{align*}
		\twoHom{\mathcal{ C } } ( X , Y ) 
		&\xhookrightarrow{f.f.}
		\twoHom{ \mathcal{ C }} ( X, Y )
		\\ 
		f
		&\mapsto
		f \circ \id_X		 
	\end{align*}
	as well as a bijection of morphism called the right unit constraint.
	\[
	\begin{tikzcd}
		f \circ \id_X
		\ar[d, Rightarrow, " \exists ! \rho_f"]
		\\
		f
	\end{tikzcd}
	\mapsto
	\begin{tikzcd}
		f \circ ( \id_X \circ \id_X )
		\ar[rr, Rightarrow, " \alpha" , " \sim"']
		\ar[rd, Rightarrow, "\id_f*v_X"' ," \sim"]
		&&
		( f \circ \id_X ) \circ \id_X  
		\ar[ld, Rightarrow, "\sim"']
		\\
		&
		f \circ \id_X 
	\end{tikzcd}
	\]
\end{construction}

\begin{prop}
	Let $ \mathcal{ C } $ be a 2-category. 
	The left and right unit constraints determine natural isomorphisms.
	\[
	\begin{tikzcd}
		\twoHom{ \mathcal{ C } } ( X , Y )
		\arrow[bend left=50]{rr}[name=U',label=above:$f\mapsto \id_Y \circ f$]{}
		\arrow[bend right=50]{rr}[name=D',label=below:$ \mathds{ 1 }$]{}
		&&
		\twoHom{ \mathcal{ C } } ( X , Y )
		\arrow[shorten <=10pt,shorten >=10pt,Rightarrow,to path={(U')-- node[label=right:$ \lambda $] {} (D')}]{}
	\end{tikzcd}
	\]
	\[
	\begin{tikzcd}
		\twoHom{ \mathcal{ C } } ( X , Y )
		\arrow[bend left=50]{rr}[name=U',label=above:$f\mapsto f \circ \id_X $]{}
		\arrow[bend right=50]{rr}[name=D',label=below:$ \mathds{ 1 }$]{}
		&&
		\twoHom{ \mathcal{ C } } ( X , Y )
		\arrow[shorten <=10pt,shorten >=10pt,Rightarrow,to path={(U')-- node[label=right:$ \rho $] {} (D')}]{}
	\end{tikzcd}
	\]
\end{prop}

Lecture 24.4

\begin{proof}
	Exercise.
	Let $ \forall \colon X \to Y, \lambda_f $ is an isomorphism.
	We only prove $ \lambda = ( \lambda_f \colon \id_Y \circ f \Rightarrow f )_{ f \in \Hom_{ \mathcal{C } } ( X , Y ) }$ is a natural transformation.
	Let 
	\[
		f \xRightarrow{\eta} g
	\]
	be a morphism in $ \Hom_{ \mathcal{ C } } ( X, Y )$ 
	\[
	\begin{tikzcd}
		\id_Y \circ f 
		\ar[d, Rightarrow, "\id_Y \circ \eta"]
		\ar[r, Rightarrow, "\lambda_f"]
		&
		f
		\ar[d, Rightarrow, "\eta"]
		\\
		\id_Y \circ g 
		\ar[r, Rightarrow, "\lambda_g"]
		& 
		g
	\end{tikzcd}
	\xrightarrow{ \id_Y \circ - }
	\begin{tikzcd}
		\id_Y \circ ( \id_Y \circ f ) 
		\ar[r,Rightarrow, "\id_Y \circ \lambda_f"]
		\ar[d, Rightarrow, "\id_Y \circ ( \id_Y \circ \eta )"']
		&
		\id_Y \circ f
		\ar[d, Rightarrow, " \id_Y \circ \eta"]
		\\
		\id_Y \circ ( \id_Y \circ g )
		\ar[r, Rightarrow]
		&
		\id_Y \circ g
	\end{tikzcd}
	\]
	
	\[
	\begin{tikzcd}
		\id_Y \circ ( \id_Y \circ f )
		\ar[ddd,Rightarrow, " \id_Y \circ ( \id_Y \circ \eta )"']
		\ar[rr, Rightarrow, " \id_Y \circ\lambda_f" ]
		\ar[rd, Rightarrow, "\sim"', "\alpha"]
		&&
		\id \circ f 
		\ar[ddd,Rightarrow, " \id_Y \circ \eta = \id_{ \id_Y } * \eta "]
		\\
		&
		(\id_Y \circ \id_Y ) \circ f 
		\ar[ru,Rightarrow, " v_y * \id_f","\sim"']
		\ar[d, Rightarrow, " ( \id_Y \circ \id_Y) \circ \eta" ]
		\\
		&
		(\id_Y \circ \id_Y ) \circ g
		\ar[rd, Rightarrow, "v_Y * \id_g", "\sim"']
		\\
		\id_Y \circ ( \id_Y \circ g ) 
		\ar[ru, Rightarrow, "\sim", " \alpha"']
		\ar[rr, Rightarrow, " \id_Y \circ \lambda_g"]	
		&&
		\id_Y \circ g
	\end{tikzcd}
	\]
	where the left square commutes by the naturality of the associator constraint, and the top and bottom triangle commute by the left unit constraint.
	For the right square we use the interchange law for composition and the Godement product to obtain $ ( \id_{\id_Y} * \eta ) \circ ( v_Y * \id_f ) = ( v_\eta * \eta ) = ( v_Y * \id_g ) \circ ( \id_Y \circ \eta )$
\end{proof}

\begin{prop}
	Let $ \mathcal{ C } $ be a 2-category and $ X \xrightarrow{f} Y \xrightarrow{g} Z $ two composable 1-morphisms in $\mathcal{ C }$.
	Then the following triangle 
	\[
	\begin{tikzcd}
		g \circ ( \id_Y \circ f)
		\ar[rr, Rightarrow, "\alpha", "\sim"']
		\ar[rd, Rightarrow,"\id_g \times \lambda_f"']
		&&
		( g \circ \id_Y ) \circ f 
		\ar[dl, Rightarrow, " \rho_g \times \id_f"]
		\\
		&
		g \circ f 
	\end{tikzcd}
	\]
	commutes.
\end{prop}

\begin{proof}
	Consider the following commutative diagram:
	\[
	\begin{tikzcd}[column sep= 0.1cm]
		& 
		g \circ ( ( \id_Y \circ \id_Y ) \circ f )
		\ar[rr, Rightarrow, "\alpha", "\sim"']
		\ar[d, Rightarrow, "v_Y"]
		&&
		( g \circ ( \id_Y \circ \id_Y ) ) \circ f 
		\ar[d, Rightarrow, "u_Y" ]
		\ar[rdd, Rightarrow, "\alpha"]
		\\
		&
		g \circ ( \id_Y \circ f )
		\ar[rr,Rightarrow, "\alpha", "\sim"']
		\ar[d, Rightarrow, "\alpha", "\sim"']
		&&
		( g \circ id_Y ) \circ f
		\\
		g \circ ( \id_Y \circ ( \id_Y \circ f ))
		\ar[ruu, Rightarrow, "\alpha"]
		\ar[ru, Rightarrow, "\lambda_f"']
		\ar[rrd, Rightarrow, "\alpha"]
		& 
		( g \circ \id_Y ) \circ f
		\ar[rru, equal]
		\ar[r, Rightarrow ,"\rho_g", " \sim"']
		&
		g \circ f 
		&
		g \circ ( \id_Y \circ f )
		\ar[ u , Rightarrow, "\alpha" , "\sim"']
		\ar[l, Rightarrow, "\lambda_f"' , "\sim"]
		&
		((g \circ \id_Y ) \circ \id_Y ) \circ f
		\ar[lu, Rightarrow, " \rho_g"]
		\\
		&&
		(g \circ \id_Y ) \circ ( \id_Y \circ f)
		\ar[lu, Rightarrow, " \lambda_f"']
		\ar[ru, Rightarrow, "\rho_g"]
		\ar[rru, Rightarrow, " \alpha"]
		\ar[u, phantom, "{*)}"]
	\end{tikzcd}
	\]
	The triangles commute by applying unit constraints  \cref{left_right_unit_constraints} and the square commute 
	and the squares that include an alpha commute by the associator constraints. 
	The only square that remains is $ *) $, here we use the interchange law for the Godement product to obtain $ ( \rho_g * \id_{f} ) ( \id_{g} * \lambda_f ) = \rho_g * \lambda_f = ( \id_{g} * \lambda_f ) \circ ( \rho_g * \id_f )$.
\end{proof}

\begin{cor}
	Let $ \mathcal{ C } $ be a 2-category and $ X \in \mathcal{ C }$, consider $ \id_X \colon X \to X $.
	Then 
	\begin{align*}
			\lambda_{ \id_X} \colon \id_X \circ \id_X 
			&\xRightarrow{\sim }
			\id_X
			\\
			\rho_{ \id_X} \colon \id_X \circ \id_X
			&\xRightarrow{\sim}
			\id_X
	\end{align*} 
	are both equal to $v_X \colon \id_X \circ \id_X \Rightarrow \id_X$.
\end{cor}

\begin{proof}
	We only do the case $ \lambda_{\id_X}= v_X $.
	By the triangle identity and definition of $ \lambda_{ \id_X} $
	we get that 
	\[
	\begin{tikzcd}	
		\id_X \circ ( \id_X \circ \id_X )
		\ar[rr, Rightarrow, "\alpha", "\sim"']
		\ar[rd, Rightarrow, "\id_X \circ \lambda_{ \id_X}"']
		&&
		(\id_X \circ \id_X ) \circ \id_X
		\ar[dl, Rightarrow," v_X * \id_X"]
		\ar[dl, Rightarrow, bend left=60, " \rho_{\id_X} * \id_X"]
		\\
		&
		\id_X \circ \id_X
	\end{tikzcd}	
	\]
	and thus $ v_X * \id_X = \rho_{\id_X} * \id_X$ which implies that $v_X = \rho_{ \id_X}$ since the composition with the identity is fully faithful.
\end{proof}

\begin{defi}
	Let $\mathcal{ C }$ be a 2-category. The conjugate of $\mathcal{ C } $ is the 2-category $ \mathcal{C}^c = \mathcal{ C }^{co} $ with $\Ob( \mathcal{ C }^c ) = \Ob ( \mathcal{ C } )$ and $ \Hom_{ \mathcal{ C }^c} ( X , Y ) \coloneqq \Hom_{ \mathcal{ C } } ( X , Y )^{ \op}.$
\end{defi}

\begin{defi}
	A $ ( 2 ,1 )$-category is a 2-category such that $ \forall X, Y \in \Ob ( \mathcal{ C } ), \Hom_{ \mathcal{ C } } (X , Y) $ is a groupoid.
\end{defi}

\begin{defi}
	Let $ \mathcal{ C } $ be a 2-category. The coarse homotopy category of $\mathcal{C}$ is the 1-ccategory $h\mathcal{C}$ with $\Ob ( \mathcal{C } ) \coloneqq \Ob(\mathcal{ C } ) $ and with sets of morphisms, $ \Hom_{ h \mathcal{ C } }  ( X , Y ) = \pi_0 ( L \underline{ \Hom }_{ \mathcal{ C } } ( X , Y )) = \pi_0 ( N \underline{\Hom}_{\mathcal{ C } }( X , Y ))$ with the induced composition law, where $L$  is the localisation functor from $ \Cat $ to $\Gpd$.
\end{defi}

\begin{defi}
	Let $\mathcal{ C }$ be a 2-category.
	The pith of $ \mathcal{ C } $ is the 2-category $ \Pith ( \mathcal{ C } )$ with objects $ \Ob ( \Pith ( \mathcal{ C } ) )=\Ob ( \mathcal{ C } )$ with $\underline{ \Hom }_{ \Pith ( \mathcal{ C } ) } ( X, Y ) \coloneqq \underline{\Hom}_{ \mathcal{ C } } ( X , Y )^{ \cong }.$
\end{defi}

\begin{defi}
	The homotopy category of $ \mathcal{ C } $ is $ \hPith ( \mathcal{ C } ) $.
\end{defi}
