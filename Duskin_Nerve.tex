\section{Duskin nerve}

The reference for this section is \cite[ch. 2.3]{kerodon}.

We fix a 2-category $ \mathcal{ C } $ that we want to construct a simplicial set from.


\begin{defi}
\label{Duskin_nerve_defi}
	The Duskin nerve is 
	\[
		\SetD
		\xleftarrow{}
		\twocatlax : N^D = u^*
	\]
	where $ \twocatlax $ is the category of 2-categories with morphisms given by strictly unital lax 2-functors.
	Then $ N^D ( \mathcal{ C } )_n \coloneqq \{ \text{strictly unital lax 2-functors } [ n ] \xrightarrow{ F } \mathcal{ C } \} $. 
\end{defi}

\begin{rmk}
\cite{kerodon}
	If $ \mathcal{C} $ is a 1-category, then $ N^D ( \mathcal{ C } ) = N ( \mathcal{ C } ) $.
	We analyze the n-simplices of $ N^D ( \mathcal{ C } ) $.
	\begin{itemize}
		\item 
		(n=0) $ [ 0 ] = \{ 0 \} \xrightarrow{ F } \mathcal{ C } $ given $ 0 \mapsto X_0 = F ( 0 ) \in \Ob( \mathcal{ C } ) $
		
		\item 
		(n=1)
		 $ [ 1 ] = \{ 0 \to 1 \} \xrightarrow{ F } \mathcal{ C } $
		 \[
		 \begin{tikzcd}
		 	0
		 	\ar[d] 
		 	\ar[r, mapsto, shift right=1.5em]
		 	&
		 	X_0
		 	\ar[d] 
		 	\ar[loop right, " f_{00} = \id_{X_0}"]
		 	\\
		 	1
		 	&
		 	X_1
		 	\ar[loop right, " f_{11} = \id_{ X_1 } " ]
		 \end{tikzcd}
		 \]
		 
		 \item 
		 (n=2)
		 $ [ 2 ] = 
		 \bigg\{
		 \begin{tikzcd}[column sep=1ex,row sep=1ex]
		 	&
		 	1
		 	\ar[rd]
		 	&
		 	\\
		 	0
		 	\ar[rr]
		 	\ar[ru] 
		 	&&
		 	2
		 \end{tikzcd}	
		 \bigg\}
		 \xrightarrow{F}
		 \mathcal{ C }
		 $
		 \[
		 \begin{tikzcd}	
		 	&
		 	X_1
		 	\ar[rd, "f_{21}"]
		 	\ar[d, Rightarrow, "\gamma"] 
		 	&
		 	\\
		 	X_0
		 	\ar[ru, "f_{10}" ]
		 	\ar[rr, "f_{20}"' ]
		 	&
		 	{}
		 	&
		 	X_2
		 \end{tikzcd}
		 \]
		 
		 \item 
		 (n=3)
		 $ [ 3 ] = 
		 \bigg\{
		 \begin{tikzcd}[column sep=1ex,row sep=1ex]
		 	&
		 	1
		 	\ar[rd]
		 	&
		 	\\
		 	0
		 	\ar[rd]
		 	\ar[rr]
		 	\ar[ru] 
		 	&&
		 	2
		 	\ar[ld]
		 	\\
		 	&
		 	3
		 	\arrow[from=uu, crossing over]
		 	&
		 \end{tikzcd}	
		 \bigg\}
		 \xrightarrow{F}
		 \mathcal{ C }
		 $
		 \[
		 \begin{tikzcd}	
		 	&
		 	X_1
		 	\ar[rd, "f_{21}"]
		 	&
		 	\\
		 	X_0
		 	\ar[rd, "f_{30}"']
		 	\ar[ru, "f_{10}" ]
		 	\ar[rr, "f_{20}"' pos=0.3 ]
		 	&
		 	{}
		 	&
		 	X_2
		 	\ar[ld, "f_{32}"]
		 	\\
		 	& 
		 	X_3
		 	\ar[from=uu, crossing over, "f_{13}" pos=0.3 ]
		 	&
		 \end{tikzcd}
		 \]
		 with a 2-morphism $ \xRightarrow{ \gamma } $ for every 2-simplex in the boundary.
		 
	\end{itemize} 

	More precisely:
	A strictly unital lax 2-functor $ F \colon [ n ] \to \mathcal{ C } $ consist of the following data
	\begin{itemize}
		\item 	
		 $ \forall 0 \leq i \leq n $ an object $ X_i \coloneqq F ( i ) $,
		
		\item   
		$ \forall 0 \leq i \leq j \leq n $ a 1-morphism $ f_{ j i } = X ( i \to j ) $,
		
		\item 
		and $ \forall 0 \leq i \leq j \leq k \leq n $ a 2-morphism $ \mu_{ k j i } = f_{ k j } \cdot f_{ j i } \Rightarrow f_{ k i } $.
	\end{itemize}

	Moreover the above data must satisfy
	
	\begin{itemize}
		\item 
		(strict unitality) $ \forall 0 \leq i \leq n , f_{ i i } = \id_{ X_i } $ the following diagram commutes
		\[
		\begin{tikzcd}
			f_{ j j } \circ f_{ j i }
			\ar[r, Rightarrow, "\mu_{ j j i }"]
			&
			f_{ j i }
			\ar[d, Rightarrow, " F ( \lambda_{ j i } ) = \id_{ f_{ j i } }"]
			\\
			\id_{ X_j } \circ f_{ j i }
			\ar[u, Rightarrow, "\id * \id"]
			\ar[r, Rightarrow, "\lambda_{ f_{ ji } } "]
			&
			f_{ j i }
		\end{tikzcd}
		\]
		that is $ \forall 0 \leq i \leq j \leq n $ 
		\begin{align*}
			\mu_{ j j i } = \lambda_{ f_{ j i } } 
			&\colon
			\id_{ X_j } \circ f_{ j i } \Rightarrow f_{ j i }
			\\
			\mu_{ j i i } = \rho_{ f_{ j i } } 
			&\colon 
			f_{ j i } \circ \id_{ X_i } \Rightarrow f_{ j i }.
		\end{align*}
		Recall that 
		\[
			\mu_{ i i i } = \lambda_{ f_{ i i } } = \id_{ X_i } = \rho_{ \id_ { X_i } } = \rho_{ f_{ i i } } = \mu_{ i i i }.
		\] 
		
		\item 
		(Composition)
		$ \forall 0 \leq i \leq j \leq k \leq l \leq n $
		
		\begin{equation}
		\label{eq:1}
		\begin{tikzcd}
			f_{lk} \circ ( f_{kj} \circ f_{ji} )
			\ar[r, Rightarrow, "\alpha"]
			\ar[d, Rightarrow, " \id * \mu"]
			&
			( f_{lk} \circ f_{kj} ) \circ f_{ji}
			\ar[d, Rightarrow, "\mu*\id"]
			\\
			f_{lk} \circ f_{ki}
			\ar[d, Rightarrow," \mu "]
			&
			f_{lj} \circ f_{ji}
			\ar[d, Rightarrow, " \mu "]
			\\
			f_{li}
			\ar[r, Rightarrow, " F ( \alpha ) = \id "]
			&
			f_{li}
		\end{tikzcd}
		\end{equation}
	\end{itemize}
\end{rmk}

\begin{prop}
	An n-simplex $ F \in N^D ( \mathcal{ C } )_n ( F \colon [ n ] \to \mathcal{ C } ) $ is uniquely determined by the following data:
	\begin{itemize}
		\item 
		$ 0 \leq i \leq n , X_i \in \Ob( \mathcal{ C } ) , \forall  0 \leq  i < j \leq n, f_j \colon X_i \to X_j $ a collection of 1-morphisms in $ \mathcal{ C } $,
		
		\item 
		$ 0 \leq i < j < k \leq n , \mu_{ k j i } = f_{ k j } \circ f_{ j i } \Rightarrow f_{ k i } $ such that $ \forall 0 \leq i < j < k < l \leq n $ \eqref{eq:1} is satisfied.  
\end{itemize}
\end{prop}

\begin{proof}
	Sketch:
	
	The uniqueness is clear, since we must have 
	\[
		\forall 0 \leq i \leq n , f_{ i i } = \id_{ X_i }, \forall 0 \leq i \leq j \leq n.
	\]
	Given the data as in the statement, we define  
	\[
		\mu_{ j j i } = \lambda_{ f_{ j i } } , \mu_{ j i i } = \rho_{ f_{ j i } }.
	\]
	We know \ref{eq:1} holds for $ i < j < l < k $.
	To check that \ref{eq:1} holds when some indices are equal is given as an Exercise, it uses the triangle identity and the following lemma.
\end{proof}

\begin{lem}		
	Let $ X \xrightarrow{ f } Y \xrightarrow{ g } Z $ be 1-morphisms in $ \mathcal{ C } $.
	Then the following diagrams commute:
	\begin{enumerate}
		\item 
		\[
		\begin{tikzcd}
			\id_Z \circ ( g \circ f ) 
			\ar[rr, Rightarrow, "\alpha"]
			\ar[rd, Rightarrow, "\lambda_{ g \circ f }"' ]
			&&
			( \id_Z \circ g ) \circ f 
			\\
			&
			g \circ f 
			\ar[ru, Rightarrow, " ( \lambda_g)^{ - 1 } * \id_f "']
			&
		\end{tikzcd}
		\]
		
		\item 
		\[
		\begin{tikzcd}	
			g \circ ( f \circ \id_X )
			\ar[rr, Rightarrow, " \alpha " ]
			\ar[rd, Rightarrow, " \id_g * \rho_f "' ]
			&&
			( g \circ f ) \id_X
			\\
			&
			g \circ f 
			\ar[ru, Rightarrow , " \rho_{ g \circ f }^{ - 1 } "' ]
		\end{tikzcd}
		\]
	\end{enumerate}
\end{lem}

\begin{proof}
	We prove 2., consider for that the following diagram, notice that we omitted taking products with identities in the notation of the morphisms. 
	\[
	\begin{tikzcd}[row sep= 1.5cm, column sep= 0.5cm]
		&
		g \circ ( ( f \circ \id_X ) \circ \id_X ) 
		\ar[r, Rightarrow, "\alpha"]
		\ar[d, Rightarrow, "\rho"]
		&
		( g \circ ( f \circ \id_X ) ) \circ \id_X 
		\ar[d, Rightarrow, "\id * \rho"] 
		\ar[rd, Rightarrow, "\alpha"]
		\\
		g \circ ( f \circ ( \id_X \circ \id_X ) ) 
		\ar[r, Rightarrow, "\lambda"]
		\ar[rrd, Rightarrow, "\alpha"']
		\ar[ru, Rightarrow, "\alpha"]
		&
		g \circ ( f \circ \id_X )
		\ar[r, Rightarrow, "\alpha"] 
		&
		( g \circ f ) \circ \id_X 
		\ar[r, Rightarrow, " \rho^{ - 1 }"]
		&
		( ( g \circ f ) \circ \id_X ) \circ \id_X 
		\\
		& &
		( g \circ f ) \circ ( \id_X \circ \id_X )
		\ar[u, Rightarrow, "\lambda"]
		\ar[ru, Rightarrow, "\alpha"']
	\end{tikzcd}
	\]
	By the pentagon axiom the outer composition in the diagram commutes, the bottom, top left and bottom right triangles commute by the triangle identity and the top middle square commutes by naturality of $ \rho $, thus the top right triangle commutes.
	Lastly the diagram we aim to show commutes is the top right triangle with $ - \circ \id_X $ applied to it, since $ - \circ \id_X $ is fully faithful the original triangle commutes.
\end{proof}

\begin{cor}
	The Duskin nerve $ N^D ( \mathcal{ C } ) $ is 3-coskeletal and 
	\[
	 	\Hom_{ \SetD } ( \Delta^3 , N^D ( \mathcal{ C } ) ) \to \Hom_{ \SetD } ( \partial \Delta^3 , N^D ( \mathcal{ C } ) ) 
	\]
  	is injective.
\end{cor}

The 3-simplices can be given by 2 boundary quadrilaterals as follows, let $ 0 \leq i < j < k < l \leq n $.
Then the boundary quadrilaterals are given by

\[
\begin{tikzcd}
	&
	X_j
	\ar[r , " f_{ k j } " ]
	&
	X_k
	\ar[rd, "f_{ l k } "]
	\\
	X_i
	\ar[rru,bend right=10,""{name=U, above}]
	\ar[ru, " f_{ j i } " ]
	\ar[rrr, " f_{ k l } "' {name=R, above}]
	\arrow[Rightarrow, from=ru, to=U, "\mu_{ k j i } "]
	&&&
	X_l
	\ar[Rightarrow, from=ul, to=R, "\mu_{ l k i } "]
\end{tikzcd}
\]
\[
\begin{tikzcd}
	&
	X_j
	\ar[r , " f_{ k j } " ]
	\ar[rrd, bend right=10,""{name=D, above}]
	&
	X_k
	\ar[rd, "f_{ l k } "]
	\ar[Rightarrow, to=D, "\mu_{ k l j } "]
	\\
	X_i
	\ar[ru, " f_{ j i } " ]
	\ar[rrr, " f_{ l i } "' {name=L, above}]
	\ar[Rightarrow, from=ur, to=L, "\mu_{ l i j } "']
	&&&
	X_l
\end{tikzcd}
\]

\begin{defi}
\label{thin_defi}
	Let $ X \in \SetD $ be a simplicial set. 
	A 2-simplex $ \sigma \in X_2 $ is thin if $ \forall n \geq 3, \forall 0 < i < n $, there exists 
	a morphism $ \Tilde{ \rho } $ for all morphisms $ \tau $ such that the following commutes
	\[
	\begin{tikzcd}
		\Delta^2 = [ 2 ] 
		\ar[r, " j "]
		\ar[rr, bend left = 60, " \sigma "]
		&
		\Lambda_i^n
		\ar[r, " \forall \tau"]
		\ar[d, hook]
		&
		X
		\\
		&
		\Delta^n 
		\ar[ru, dashed, "\exists \Tilde{\rho}"']
	\end{tikzcd}
	\] 	
	with $ j \coloneqq [ 2 ] \to \{ i -1 , i , i + 1 \} $.
\end{defi}

\begin{rmk}
\label{every_simplex_thin}
	Let $ X \in \SetD $ be an $ \infty $-category, then every 2-simplex of $ X $ is thin.
	Conversely, if every 2-simplex of $ X $ is thin, then $ X $ is an $ \infty $ category if and only if the following diagram commutes:
	\[
	\begin{tikzcd}
		\Lambda_1^2 
		\ar[r, " \forall"]
		\ar[d, hook]
		&
		X
		\\
		\Delta^2
		\ar[ru, dashed, "\exists"']
	\end{tikzcd}
	\]
\end{rmk}

\begin{rmk}
\label{remark_thin_Duskin}
	Take the following horn filling problem
	\[
	\begin{tikzcd}
		\Lambda_1^2
		\ar[r, " \sigma "]
		\ar[d, hook]
		&
		N^D ( \mathcal{ C } )
		\\
		\Delta^2
		\ar[ru, dashed, "\exists"']
	\end{tikzcd}
	\]
	where the dashed arrow exists since we have the horizontal composition of 1-morphisms.
	Thus we can extend any 2-horn and would have an infinity category if every 2-simplex were thin.
\end{rmk}

\begin{thm}
	Let $ \mathcal{ C } $ be a 2-category, then $ \mathcal{ C } $ is a $ ( 2 , 1 ) $-category if and only if $ N^D ( \mathcal{ C } ) $ is an $ \infty $-category.
\end{thm}

\begin{proof}
	Suppose $ \mathcal{ C } $ is a  (2,1)-category.
	Assume every 2-simplex were thin by \ref{thin_iff_invertible}, then by \ref{remark_thin_Duskin} $ N^D ( \mathcal{ C } ) $ is an $ \infty $-category.
	Conversely, if $ N^D (\mathcal{ C } ) $ is an $ \infty $-category, then by \ref{every_simplex_thin} every 2-simplex in $ N^D ( \mathcal{ C } ) $ is thin if the following is satisfied.
	Let  
	\[
	\begin{tikzcd}
		X
		\arrow[bend left=50]{rr}[name=U',label=above:$f$]{}
		\arrow[bend right=50]{rr}[name=D',label=below:$j$]{}
		&&
		Y
		\arrow[shorten <=10pt,shorten >=10pt,Rightarrow,to path={(U')-- node[label=right:$ \alpha $] {} (D')}]{} 
	\end{tikzcd}
	\]
	be a 2-morphism in $ \mathcal{ C } $, then we obtain a commutative diagram
	\[
	\begin{tikzcd}	
		\id_Y \circ f 
		\ar[r, Rightarrow, "\lambda_f"]
		\ar[rd, Rightarrow, "\rho"' ]
		&
		f
		\ar[d, Rightarrow, " \gamma "]
		\\
		&
		g
	\end{tikzcd}
	\]
	where $\rho $ exists since $ N^D ( \mathcal{ C } ) $ is thin.
	Now by assumption $ \rho $ is invertible, thus $ \rho $ is invertible.
\end{proof}

Lecture 27.5

For the next part let $ \mathcal{ C } $ be a fixed 2-category and let $ N^D ( \mathcal{ C } ) \in \SetD $ be its Duskin nerve, i.e. $ N^D ( \mathcal{ C } )_n \coloneqq \{ [ n ] \to \mathcal{ C } \mid \text{strictly unital lax functors} \} $.
	
\begin{thm}
\label{thin_iff_invertible}
	The 2-simplex 
	$
	\begin{tikzcd}	
		&
		X_1
		\ar[rd, "f_{21}"]
		\ar[d, Rightarrow, "\gamma_{ 2 1 0 }"] 
		&
		\\
		X_0
		\ar[ru, "f_{10}" ]
		\ar[rr, "f_{20}"' ]
		&
		{}
		&
		X_2
	\end{tikzcd}
	\in
	N^D ( \mathcal{ C } )_2 
	$
	is thin  if and only if $ \mu_{ 2 1 0 } $ is invertible.
\end{thm}

\begin{proof}
	We divide the proof into its natural parts.
\end{proof}
	
\begin{prop}
	Let $ n \geq 3 $ as well as $ 0 \leq l \leq n $, 
	\[
	\sigma
	=
	\begin{tikzcd}	
		&
		X_1
		\ar[rd, "f_{21}"]
		\ar[d, Rightarrow, "\gamma"] 
		&
		\\
		X_0
		\ar[ru, "f_{10}" ]
		\ar[rr, "f_{20}"' ]
		&
		{}
		&
		X_2
	\end{tikzcd}
	\]
	be a 2-simplex in $ N^D ( \mathcal{ C } ) $ and 
	\[
	\begin{tikzcd}
		\Lambda_l^n 
		\ar[r, "u"]
		&
		N^D ( \mathcal{ C } )
		\\
		\Delta^{ \{ l - 1 , l , l + 1 \} }
		\ar[ru, " \sigma "]
		\ar[u]
	\end{tikzcd}
	\]
	commute.
	If $ \gamma $ is invertible, then $ u $ extends uniquely to an $ n $-simplex of $ N^D ( \mathcal{ C } ) $.
 \end{prop}

\begin{proof}
	Recall that $ N^D ( \mathcal{ C } ) $ is 3-coskeletal, hence we may assume $ n = 3 , 4 $.
	(Case 1) $ n = 3 , l = 1 $ 
	\[
	\begin{tikzcd}
		&
		X_1
		\ar[r , " f_{ 2 1 } " ]
		&
		X_2
		\ar[rd, "f_{ 3 2 } "]
		\ar[rd, green, dash, dashed, shift right]
		\\
		X_0
		\ar[rrr, green, dash, dashed, shift left,
			start anchor={[xshift=2ex]}]
		\ar[rru,bend right=10,""{name=U, below}]
		\ar[rru, green, dash, dashed, shift right, bend right=10, 
		start anchor={[xshift=2ex]},
		start anchor={[yshift=0.5ex]}]
		\ar[ru, " f_{ 1 0 } " ]
		\ar[rrr, " f_{ 3 0 } "']
		\arrow[Rightarrow, from=ru, to=U, shorten=3, "\mu_{ 2 1 0 } = \gamma"]
		&&&
		X_3
	\end{tikzcd}
	\quad
	\begin{tikzcd}
		&
		X_1
		\ar[r , " f_{ 2 1 } " ]
		\ar[rrd, bend right=10,""{name=D, below}]
		&
		X_2
		\ar[rd, "f_{ 3 2 } "]
		\ar[Rightarrow, to=D, shorten = 3, "\mu_{ 3 2 1 } "]
		\\
		X_0
		\ar[ru, " f_{ 1 0 } " ]
		\ar[rrr, " f_{ 3 0 } "' {name=L, below}]
		\ar[Rightarrow, from=ur, to=L, shorten = 3, "\mu_{ 3 1 0 } "']
		&&&
		X_3
	\end{tikzcd}
	\]
	Where the dashed green lined simplex corresponds to the missing simplex given by $ \mu_{ 3 2 0 } $, which we can construct as follows.
	Observe that since $ \id_{ f_ { 3 2 } } $ and $ \mu_{ 2 1 0 } = \gamma $ are invertible and composition is functorial, the composition is invertible as well. 
	%# reference for Exercise missing
	We know from an Exercise that if $ \mu_{ 3 2 0 } $ would exist, the following identity would hold and that it is sufficient that this holds to extend the horn
	\[
		\mu_{ 3 2 0 } ( \id_{ f_ { 3 2 } }  \circ \mu_{ 2 1 0 } ) 
		=
		\mu_{ 3 1 0 } ( \mu_{ 3 2 1 } \circ \id_{ f_{ 1 0 } } ) \alpha_{ f_{ 3 2 } , f_{ 2 1 } , f_{ 1 0 } } .
	\]
	By the above argument we can choose
	\[
		\mu_{ 3 2 0 } 
		=
		( \mu_{ 3 1 0 } ( \mu_{ 3 2 1 } \circ \id_{ f_{ 1 0 } } ) \alpha_{ f_{ 3 2 } , f_{ 2 1 } , f_{ 1 0 } } )  ( \id_{ f_ { 3 2 } } \circ \mu_{ 2 1 0 } )^{ - 1 }.
	\]
	\newline
	(Case 2: n=4 ,l=2) $ u \colon \Lambda_l^n \to N^D ( \mathcal{ C } ) $
	\newline
	We need to check the compatibility of composition, since all other diagrams for the Duskin nerve require only simplices up dimension three to commute and, thus hold for the boundary 3-simplices of a 4-horn, leaving only the compatibility of composition to be checked.
	The outer square of the following diagram is exactly the compatibility of composition diagram for the missing 3 simplex, i.e. all morphisms that do not contain a $ 2 $ as an index.
	\[
	\begin{tikzcd}[column sep=0.7]
		f_{ 4 3 } \circ ( f_{ 3 1 } \circ f_{ 1 0 } )
		\ar[rrrr, Rightarrow, "\alpha"]
		\ar[ddddd, Rightarrow, " \mu_{ 3 1 0 } "]
		&&&&
		( f_{ 4 3 } \circ f_{ 3 1 } ) \circ f_{ 1 0 }
		\ar[ddddd, Rightarrow, " \mu_{ 4 3 1 } "]
		\\
		&
		f_{ 4 3 } \circ ( ( f_{ 3 2 } \circ f_{ 2 1 } ) \circ f_{ 1 0 } )
		\ar[rr, Rightarrow, "\alpha"]
		\ar[lu, Rightarrow,  "\mu_{ 3 2 1 }"]
		&&
		( f_{ 4 3 } \circ ( f_{ 3 2 } \circ f_{ 2 3 } ) ) \circ f_{ 1 0 } 
		\ar[ru, Rightarrow]
		\ar[d, Rightarrow, "\alpha"]
		\\
		&
		f_{ 4 3 } \circ ( f_{ 3 2 } \circ ( f_{ 2 1 } \circ f_{ 1 0 } ) )
		\ar[d, Rightarrow, "\mu_{ 2 1 0 }"]
		\ar[rd, Rightarrow, "\alpha"]
		\ar[u, Rightarrow, "\alpha"]
		&&
		( ( f_{ 4 3 } \circ f_{ 3 2 } ) f_{ 2 1 } ) f_{ 1 0 }
		\ar[d, Rightarrow, "\mu_{ 4 3 2 }"]
		\\
		&
		f_{ 4 3 } \circ ( f_{ 3 2 } \circ f_{ 2 0 } )
		\ar[ddl, Rightarrow, "\mu_{ 3 2 0 }"']
		\ar[d , Rightarrow, " \alpha"]
		&
		( f_{ 4 3 } \circ f_{ 3 2 } ) \circ ( f_{ 2 1 } \circ f_{ 1 0 } )
		\ar[dl, Rightarrow, "\mu_{ 2 1 0 }"]
		\ar[dr, Rightarrow, "\mu_{ 4 3 2 }"]
		\ar[ru, Rightarrow, "\alpha"]
		&
		( f_{ 4 2 } \circ f_{ 2 1 } ) \circ f_{ 1 0 } 
		\ar[d, Rightarrow, "\alpha"]
		\ar[ddr, Rightarrow, " \mu_{ 4 2 1 }"]
		\\
		&
		( f_{ 4 3 } \circ f_{ 3 2 } ) \circ f_{ 2 0 }
		\ar[r, Rightarrow, "\mu_{ 4 3 2 }"']
		&
		f_{ 4 2 } \circ f_{ 2 0 } 
		\ar[d, Rightarrow, " \mu_{ 4 2 0 } "]
		&
		f_{ 4 2 } \circ ( f_{ 2 1 } \circ f_{ 1 0 } )
		\ar[l, Rightarrow, "\mu_{ 2 1 0 }"] 
		\\
		f_{ 4 3 } \circ f_{ 3 0 }
		\ar[rr, Rightarrow, "\mu_{ 4 3 0 } "]
		&&
		f_{ 4 0 }
		&&
		f_{ 4 1 } \circ f_{ 1 0 } 
		\ar[ll, Rightarrow, "\mu_{ 4 1 0 }"]
	\end{tikzcd}
	\]
	Note that composition constraint and associativity constraint are isomorphisms, thus if the inner diagrams commute, so does the outer one.
	The left and right quadrilaterals commute by the compatibility of composition, given by the other boundary 3-simplices of the horn and since composition with a morphism ($ f_{ 4 3 } $ on the left side, $f_{ 4 2 } $ on the right side) is fully faithful, the upper square commutes by the naturality of $ \alpha $, the pentagon by the Pentagon identity, the triangles below the Pentagon by the naturality of the associativity constraint, the bottom left and right square commute by the composition compatibility.
	%# one squares commutativity missing here
\end{proof}

\begin{prop}
	Let $ \mathcal{ C } $ be a 2-category and 
	\[
	\sigma \coloneqq
	\begin{tikzcd}
		& 
		X_1
		\ar[rd,"g"]
		\ar[d, Rightarrow, "\gamma"]
		\\
		X_0 
		\ar[ru, "f"]
		\ar[rr, "h"']
		&
		{}
		&
		X_2
	\end{tikzcd}
	\in N^D (\mathcal{ C } )_2
	\]
	such that $ \forall n \in \{ 3 , 4 \} $ the following diagram commutes
	\[
	\begin{tikzcd}
		\Delta^{ \{ 0 , 1 , 2 \} }
		\ar[d, hook]
		\ar[rd, bend left, "\sigma"]
		\\
		\Lambda_1^n
		\ar[r, "u"]
		\ar[d, hook]
		&
		N^D ( \mathcal{ C } )
		\\
		\Delta^n
		\ar[ru, "\exists"]
	\end{tikzcd}
	\]
	then $ \gamma $ is invertible.
\end{prop}

\begin{proof}
	Let the following quadrilaterals be the boundary of a 3-simplex.
	\[
	\begin{tikzcd}
		&
		X_1
		\ar[r , " f " ]
		&
		X_2
		\ar[rd, " g "]
		\ar[rd, green, dash, dashed, shift right]
		\\
		X_0
		\ar[rrr, green, dash, dashed, shift left,
		start anchor={[xshift=2ex]}]
		\ar[rru,bend right=10," h "'{name=U, below}]
		\ar[rru, green, dash, dashed, shift right, bend right=10, 
		start anchor={[xshift=2ex]},
		start anchor={[yshift=0.5ex]}]
		\ar[ru, " f " ]
		\ar[rrr, " g \circ f "']
		\arrow[Rightarrow, shorten=3, from=ru, to=U, "\gamma"]
		&&&
		X_3 = X_2
	\end{tikzcd}
	\quad
	\begin{tikzcd}
		&
		X_1
		\ar[r , " g " ]
		\ar[rrd, bend right=10,shorten=3," g "{name=D, below}]
		&
		X_2
		\ar[rd, "\id_{ X_2 } "]
		\ar[Rightarrow, to=D,shorten=3, " \id_g "]
		\\
		X_0
		\ar[ru, " f " ]
		\ar[rrr, " g \circ f " {name=L, below}]
		\ar[Rightarrow,shorten=3, from=ur, to=L, "\id_{ g \circ f } "']
		&&&
		X_3 = X_2
	\end{tikzcd}
	\]
	by assumption this has a filling and thus there exists a $ \delta $ such that $ \delta \gamma = \id_{ g \circ f } \circ \id_g $, we can assume strict unitality of the category (for details on this see \cite{kerodon}) and thus need obtain that $ \delta \gamma = \id_{ g \circ f } $. 
	Now we need to show it is a right inverse to $ \gamma $ as well. 
	for that we take the 4-horn given by the following boundary data:
	\[
	\begin{tikzcd}
		&& 
		X_2 
		\ar[rd, " \id_{ X_2 } "]
		\\
		&
		X_1
		\ar[ru, "g"]
		&&
		X_3 = X_2
		\ar[rd, " \id_{X_2} "]
		\\
		X_0 
		\ar[ru, " f "]
		\ar[rrrr, " h "']
		\ar[rruu, bend right, " h "']
		\ar[rrru, bend right = 10, " g \circ f "' pos = 0.8]
		&&&&
		X_4 = X_2
	\end{tikzcd}
	\]
	\[
	\begin{tikzcd}
		&& 
		X_2 
		\ar[rd, " \id_{ X_2 } "]
		\ar[rrdd, bend right=30, " \id_{ X_2 } "' pos = 0.2]
		\\
		&
		X_1
		\ar[rrrd, bend right = 10, " g "' pos = 0.3]
		\ar[ru, "g"]
		&&
		X_3 = X_2
		\ar[rd, " \id_{X_2} "]
		\\
		X_0 
		\ar[ru, " f "]
		\ar[rrrr, " h "']
		&&&&
		X_4 = X_2
	\end{tikzcd}
	\]
	\[
	\begin{tikzcd}
		&& 
		X_2 
		\ar[rd, " \id_{ X_2 } "]
		\\
		&
		X_1
		\ar[rr, " g "' pos = 0.3]
		\ar[ru, "g"]
		&&
		X_3 = X_2
		\ar[rd, " \id_{X_2} "]
		\\
		X_0 
		\ar[ru, " f "]
		\ar[rrrr, " h "']
		\ar[rrru, "g \circ f"']
		&&&&
		X_4 = X_2
	\end{tikzcd}
	\]
	\[
	\begin{tikzcd}
		&& 
		X_2 
		\ar[rd, " \id_{ X_2 } "]
		\\
		&
		X_1
		\ar[rr, " g " pos = 0.3]
		\ar[ru, "g"]
		\ar[rrrd, "g "']
		&&
		X_3 = X_2
		\ar[rd, " \id_{X_2} "]
		\\
		X_0 
		\ar[ru, " f "]
		\ar[rrrr, " h "']
		&&&&
		X_4 = X_2
	\end{tikzcd}
	\]
	\[
	\begin{tikzcd}
		&& 
		X_2 
		\ar[rd, " \id_{ X_2 } "]
		\ar[rrdd, bend right = 20, " \id_{ X_2 } "']
		\\
		&
		X_1
		\ar[ru, "g"]
		&&
		X_3 = X_2
		\ar[rd, " \id_{X_2} "]
		\\
		X_0 
		\ar[rruu, bend right, "h"']
		\ar[ru, " f "]
		\ar[rrrr, " h "']
		&&&&
		X_4 = X_2
	\end{tikzcd}
	\]
	Since the horn given by the simplices above extends, we obtain $\gamma \delta = \id_h $. 
\end{proof}

\subsection{Exercises}

\begin{Exercise}
	Given a 2-category $ \mathcal{ C } $, we defined in the lecture its Duskin Nerve to have as n-simplices the strictly unital lax functors $ [ n ] \to \mathcal{ C } $.
	Hence an n-simplex $ X \colon [ n ] \to \mathcal{ C } $ consists of a choice of objects $ \{ X_i \}_{ 0 \leq i \leq n } $ and 1-morphisms $ f_{ j , i } \colon  X_i \to X_j $ for $ 0 \leq i \leq j \leq n $ and the composition constraint, i.e. 2-morphism $ \mu_{ k , j , i } \colon f_{ k , j } \circ f_{ j , i } \to f_{ k , i } $.
	We assume the identity constraint to be the identity, we must have $ f_{ i, i }  = \id_{ X_i } $ for $ 0 \leq i \leq n $ and $ \mu_{ j , j , i } = \lambda_{ f_{ j , i } } $ and $ \mu_{ j , i , i } = \rho_{ f_{ j , i } } $ for $ 0 \leq i \leq j \leq n$.
	Furthermore, the composition constraint must satisfy
	\[
	\mu_{ l , k , i } \cdot ( \id_{ f_{ l , k } } * \mu_{ k , j , i } ) = \mu_{ l , j , i } \cdot ( \mu_{ l , k , j } * \id_{ f_{ j , i } } ) \cdot \alpha_{ f_{ l , k } , f_{ k , j } , f_{ j , i } }
	\]
	for all $ 0 \leq i \leq j \leq k \leq l \leq n $.
	
	\begin{enumerate}[label=(\alph*)]
		\item 
		Argue that in the above $ \mu_{ i , i , i } $ is well defined for $ 0 \leq i \leq n $ and if $ j = k $, then the composition constraint is automatically fulfilled.
		
		\item 
		Show that in any 2-category we have that 
		\[
		\alpha_{ \id_z , g , f } = ( \lambda_g^{ - 1 } * \id_f ) \cdot \lambda_{ g \circ f } \quad \text{ and } \quad \alpha_{ g , f , \id_x } = \rho_{ g \circ f }^{ - 1 } \cdot ( \id_g * \rho_f ) 
		\]
		for any two 1-morphism $ f \colon x \to y $ and $ g \colon y \to z $ in $ \mathcal{ C } $.
		
		\item 
		Deduce, that it suffices to require
		\[
		\mu_{ l , k , i } \cdot ( \id_{ f_{ l , k } } * \mu_{ k , j , i } ) 
		=
		\mu_{ l , j , i } \cdot ( \mu_{ l , k , j } * \id_{ f_{ j , i } } ) \cdot 
		\alpha_{ f_{ l , k } , f_{ k , j } , f_{ j , i } } 
		\]
		for all $ 0 \leq i < j < k < l \leq n $ in the description of the n-simplices.
		
		\item 
		Conclude with the help of Exercise 6.3 that the Duskin nerve is 3-coskeletal.
	\end{enumerate}
\end{Exercise}

\begin{Exercise}
	Let $ P = ( P , \leq ) $ be a partially ordered set. 
	The goal of this exercise is to show that strictly unit lax functors $ P \to \mathcal{ C } $ for a strict $ 2 $-category $ \mathcal{ C } $ are in bijection to strict functors from the path 2-category $ \Path ( P ) $.
	Here $ \Path ( P ) $ is the strict 2-category with the same objects as $ P $ and for two objects $ x , y \in P $ their morphism category $ \intHom_{ \Path ( P ) } ( x , y ) $ is given by the opposite category of finite totally ordered subsets $ S \subset P $ with $ \min ( S ) = x $ and $ \max ( S ) = y $ ordered by inclusion.
	The horizontal composition is given by the union with the singletons $ \{ x \} $ serving as identities.
	
	\begin{enumerate}[label=(\alph*)]
		\item 
		Compute the homotopy and coarse homotopy category of $ \Path ( P ) $ from Exercise 4.4.
		Which one should be called the path (1-)category of $P$?
		
		\item 
		Show that there is a unique strictly unital lax functor $ T_P \colon P \to \Path ( P ) $ which is the identity on objects and sends the 1-morphism $ x \leq y $ to the final element of $ \Hom_{ \Path ( P ) } ( x , y ) $.
		
		\item 
		Show that a strict 2-functor $ F \colon \Path ( P ) \to \mathcal{ C } $ is completely determined by $ f_{ y , x } \coloneqq F ( \{ x \leq y \} ) $ and 
		$ \mu_{ z , y , x } \coloneqq F ( \{ x \leq y \leq z \} ) \supseteq \{ x \leq z \} ) $ and that any such choice of $ \{ f_{ y ,x } \}_{ x ,y \in P } $ and $ \mu $ determines a unique functor if and only if $ \mu $ satisfies the associativity constraint of a strictly unital functor $ \Tilde{ F } \colon P \to \mathcal{ C } $.
		
		\item 
		Conclude that precompostion with $ T_P $ induces a bijection 
		\[
		( T_P )^* \colon \{ \text{ strict functors } \Path ( P ) \to \mathcal{ C } \}
		\leftrightarrow \{ \text{ strictly unital lax functors } P \to \mathcal{ C } \}.
		\]
		
		\item 
		Deduce that $ \Path $ upgrades to a functor from the category of partially ordered sets to the category of strict 2-categories with strict functors. 
		Use this to give an alternative description of the Duskin Nerve of a strict 2-category.
	\end{enumerate}
\end{Exercise}

\begin{Exercise}
	A functor $ F \colon \Delta^{ \op } \to \mathcal{ C } $ is called a simplicial object in $ \mathcal{ C } $ where we often identify the functor with objectwise images $ F_n \coloneqq F ( [ n ] ) $ and the images of the face maps $ \partial_i^n \coloneqq F ( d_i^n ) \colon F_n \to F_{ n - 1 } $ and degeneracy maps $ \sigma_i^n \coloneqq F ( s_i^n ) \colon F_n \to F_{ n + 1 } $.
	Recall that we thus have in particular that for the face maps and degeneracy maps obey the following relations
	\begin{align*}
		\partial_i^{ n - 1 } \partial_j^n = \partial_{ j - 1 }^{ n - 1 } \partial_i^n \quad \text{if } i < j
		\\
		\sigma_i^{ n + 1 } \sigma_j^n = \sigma_{ j + 1 }^{ n + 1 } \sigma_i^n \quad \text{if } i \leq j
	\end{align*}
	\[
	\partial_i^{ n + 1 } \sigma_j^n =
	\begin{cases}
		\sigma_{ j - 1 }^{ n - 1 } \partial_i^n & \text{ if } i < j 
		\\
		\id_{ F_n } & \text{ if } i \in \{ j , j + 1 \} 
		\\
		\sigma_{ j }^{ n - 1 } \partial_{ i - 1 }^n & \text{ if } i > j +1
	\end{cases}
	\]
	and that any morphism in the image of $ F $ can be written as a composition of face maps followed by degeneracy maps.
	Let now $ A \colon \Delta^{ \op } \to \Ab $ be a simplicial abelian group.
	For $ n \in \mathbb{ N }_0 $ the face maps assemble into a morphism $ \partial^n \coloneqq \sum_{ i = 0 }^n ( - 1 )^i \partial_i^n \colon A_n \to A_{ n - 1 } $.
	\begin{enumerate}[label=(\alph*)]
		\item 
		Show that $ C_{ \bullet } ( A ) \coloneqq ( A_n , \partial^n )_{ n \in \mathbb{ N }_0 } $ is an $ \mathbb{ N }_0 $-graded chain complex, i.e. $ \partial^n \partial^{ n + 1 } = 0 $.
		Furthermore, show that $ D_{ \bullet } ( A ) \coloneqq \sum_{ 0 \leq i \leq n \in \mathbb{ N }_0 } \im ( \sigma_i^n ) \subseteq C_{ \bullet } ( A ) $ is a subcomplex.
		
		\item 
		Argue that both constructions $ C_{ \bullet } $ and $ D_{ \bullet } $ are functorial on the category of simplicial groups.
		Conclude that there is a functor $ N_{ \bullet } \coloneqq C_{ \bullet } / D_{ \bullet } $.
	\end{enumerate}
	
	The complex $ N_{ \bullet } ( A ) $ is called the normalized Moore complex of $ A $.
	Recall that the forgetful functor $ \Ab \to \Set $ admits a left adjoint $ Fr \colon \Set \to \Ab $ which associates to a set $ S $ the free group $ \mathbb{ Z } S $. 
	The normalised Moore complex of a simplicial set $ X $ is then defined to be $ N_{ \bullet } ( X , \mathbb{ Z } ) \coloneqq N_{ \bullet } \circ Fr_* ( X ) = N_{ \bullet } ( Fr \circ X ) $.
	
	\begin{enumerate}[resume, label=(\alph*)]
		\item 
		Argue that for a simplicial set $ X $, its normalised Moore complex $ N_{ \bullet } ( X ) $ is level wise a free abelian group whose basis are the non-degenerate simplices.
		Describe the differential of $ N_{ \bullet } ( X ) $ in terms of this basis.
	\end{enumerate}
\end{Exercise}